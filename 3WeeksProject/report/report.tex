% This is the main report file

\documentclass{acm_proc_article-sp}
\usepackage[utf8]{inputenc}
\usepackage{listings}
\begin{document}

\title{02321 Hardware/Software programming Jan 11}
\subtitle{[Technical University of Denmark]
%\titlenote{This report should also be available online at \texttt{www.retrospekt.dk/02228report}}
}

\numberofauthors{2}
\author{
\alignauthor 
Kim Rostgaard Christensen\\
       \email{s084283@student.dtu.dk}
\alignauthor 
Morten Hillebo (s072923)
       \email{s072923@student.dtu.dk}
}


\maketitle

\begin{abstract}
%Abstract; A brief summary of all of the report including the conclusion section
%but excluding the acknowledgements, references and any appendixes.

\end{abstract}

\section{Introduction}
\label{sec:introduction}

%ACKNOWLEDGMENTS are optional
%\section{Acknowledgments}
%\input{acknowledgements.tex}
%Acknowledgments; Acknowledge any persons important to the work.

%References; A list of reference material used. All material must be cited in the
%text.

\section{Memory mapping}
In order to provide access to the various I/O devices available to the processor, an interface is needed. This is in the form of a memory map, where certain areas of the memory is significant - or reserved. As the name would suggest, these areas are mapped onto the registers of the corresponding devices. A list of the I/O mappings can be found in table \ref{table:io_mappings}

\begin{table}[h]
\centering
    \begin{tabular}{ | l | l | l |}
    \hline
     Register & Memory address \\ \hline 
    \hline
    Stdin Status Register  & xFE00  \\ \hline
    Stdin Data Register    & xFE02  \\ \hline
    Stdout Status Register & xFE04  \\ \hline
    Stdout Data Register   & xFE06  \\ \hline
    Switches Data Register & xFE0A  \\ \hline
    Buttons Data Register  & xFE0E  \\ \hline
    7SegDisplay Data Register  & xFE12  \\ \hline
    Leds Data Register  & xFE16  \\ \hline
    \end{tabular}
\caption{I/O mappings of the LC3}
\label{table:io_mappings}
\end{table}

Citations: \cite{patt2000introduction} \cite{chu2008fpga}

\appendix

% The following two commands are all you need in the
% initial runs of your .tex file to
% produce the bibliography for the citations in your paper.
\bibliographystyle{plain}
%\nocite{*}
\bibliography{sigproc}  % sigproc.bib is the name of the Bibliography in this case
% You must have a proper ".bib" file
%  and remember to run:
% latex bibtex latex latex
% to resolve all references
%
% ACM needs 'a single self-contained file'!
%
%APPENDICES are optional


%Appendixes; Appendixes holds, for example, results or figures that are not
%relevant to place in the body of the report. Appendixes should generally be
%avoided and might not be read by the course staff.


\end{document}
