% This is the main report file

\documentclass{acm_proc_article-sp}
\usepackage[utf8]{inputenc}
\usepackage{listings}
\begin{document}

\title{02321 Hardware/Software programming Jan 11}
\subtitle{[Technical University of Denmark]
%\titlenote{This report should also be available online at \texttt{www.retrospekt.dk/02228report}}
}

\numberofauthors{2}
\author{
\alignauthor 
Kim Rostgaard Christensen\\
       \email{s084283@student.dtu.dk}
\alignauthor 
Morten Hillebo (s072923)
       \email{s072923@student.dtu.dk}
}

\maketitle

\begin{abstract}
%Abstract; A brief summary of all of the report including the conclusion section
%but excluding the acknowledgements, references and any appendixes.
This project will cover the implementations of the LC-3 computer and the classic video game Snake to a FPGA-board. 
The goal though out the project has been to implement all aspects of the computer and the video game on the FPGA-board so no help was needed by additional computer resources.
\end{abstract}

\section{Introduction}
\label{sec:introduction}

%TODO explain what the snake game is - briefly
The video snake game was first released in the mid 1970's and has since become a true video game classic. The game play involves navigating a snake looking thing around on the screen in search for food, that will make the snake gain body length and increment the game score. 
The walls and hurdles in the play area will cause the snake to die if they collide.
This project is about developing and implementing the snake game on the LC3 processor. 
%ACKNOWLEDGMENTS are optional
%\section{Acknowledgments}
%\input{acknowledgements.tex}
%Acknowledgments; Acknowledge any persons important to the work.

%References; A list of reference material used. All material must be cited in the
%text.
\section{Requirement specification}
The snake game requirement specification will be split in two sections, the minimal game implementation and the enhancements. 

\subsection{Minimal implementation}
The minimal implementation requirements was specified as below in arbitrary order.    
\begin{itemize}
\item Basic game logic (snake control, food consumption and growth)
\item Snake control via serial console 
\item Basic game display (level and snake by blocks) 
\item Simple levels
\end{itemize}

\subsection{Enhancements}
For the enhancements following improvement was specified also in arbitrary order.
\begin{itemize}
\item Control by the keyboard on the LC3 board 
\item Enhanced game display (sprites instead of blocks) 
\item Enhance game play various objects (power-ups)  
\item Complex levels with obstacles
\item Multi player functionality 
\end{itemize}

\section{Design}
There are many different options when it comes to the implementation of the game. The game can, for instance, run partially on a PC or completely on the LC3 - we have aimed for the latter.
\subsection{Memory mapping}
In order to provide access to the various I/O devices available to the processor, an interface is needed. This is in the form of a memory map, where certain areas of the memory is significant - or reserved. As the name would suggest, these areas are mapped onto the registers of the corresponding devices. A list of the I/O mappings can be found in table \ref{table:io_mappings}

\begin{table}[h]
\centering
    \begin{tabular}{ | l | l | l |}
    \hline
     Register & Memory address \\ \hline 
    \hline
    Video memory           & xE000  \\ \hline
    Stdin Status Register  & xFE00  \\ \hline
    Stdin Data Register    & xFE02  \\ \hline
    Stdout Status Register & xFE04  \\ \hline
    Stdout Data Register   & xFE06  \\ \hline
    Switches Data Register & xFE0A  \\ \hline
    Buttons Data Register  & xFE0E  \\ \hline
    7SegDisplay Data Register  & xFE12  \\ \hline
    Leds Data Register  & xFE16  \\ \hline
    \end{tabular}
\caption{I/O mappings of the LC3}
\label{table:io_mappings}
\end{table}
Notice that the video memory is not a single address, but an address range. The first incarnation of the video memory used a classical write-only memory, but it became apparent when software was written that this was impractical in regards of house keeping. We realized that we could use the benefit of having complete control of the system to simply to software implementation. This became the birth of the hardware game model.
\subsubsection{Hardware game model}

Citations: \cite{patt2000introduction} \cite{chu2008fpga}

\appendix

% The following two commands are all you need in the
% initial runs of your .tex file to
% produce the bibliography for the citations in your paper.
\bibliographystyle{plain}
%\nocite{*}
\bibliography{sigproc}  % sigproc.bib is the name of the Bibliography in this case
% You must have a proper ".bib" file
%  and remember to run:
% latex bibtex latex latex
% to resolve all references
%
% ACM needs 'a single self-contained file'!
%
%APPENDICES are optional

\section{Report distribution}
Kim Rostgaard Christensen
\begin{itemize}
\item stuff
\end{itemize}

Morten Hillebo
\begin{itemize}
\item stuff
\end{itemize}

%Appendixes; Appendixes holds, for example, results or figures that are not
%relevant to place in the body of the report. Appendixes should generally be
%avoided and might not be read by the course staff.


\end{document}
