The solution is done by dividing the state machine into a FSMD as shown in figure~\ref{fig:fsmd} in appendix~\ref{app:figures} and attaching a datapath

\subsection{The individual states}
we are using nine different states for our logic, each of them implemented in order to fulfill the requirements.
\subsubsection{reset}
The first one is the reset state, where we start from on power on and go to on a hard reset. This state takes care of setting the up the cost and
then "count up" the internal coin counters CC1 and CC2. In a real machine this could be done by a seperate processor or circuit (weighing all coins
and deviding by the weight of one coin) outputting this count on a bus. This processor/circuit could also be shut down in all other states but the 
reset state to preserve power. This state automatically jumps to init on the next clock rising edge.

\subsubsection{init}
This is the initialization state that takes care of everything that has to be done prior to a purchase. This goes directly to the purchase state on the next clock.

\subsubsection{purchase}
This is the main state for doing transactions. The user operating the vending machine is being put back to this state as long as he/she has not inserted enough
money. The state also responds to coin inputs and goes to the relevant states (add1 and add2). When the return\_change button is pressed, we go
to the return\_all state corresponding to requirement 8. We also go to this state if the machine is unable to return change. This is detected by
the datapath and given by the boolean expression show on the transition between purchase and return\_all in figure~\ref{fig:fsmd} (appendix~\ref{app:figures}). This corresponds to requirement 10. Note that we in the FSMD also check if TCC1 is zero. This enables us to detect if 
a user has inserted 8 kr using 2,4 or 6 1 kr and ending with a 2 kr. In this case the machine actually holds change, but we need to do more in order
to make this work - to be continued.\\
\\
When the user has inserted enough money, we move to the merge\_counters state. The logic is shown in figure~\ref{fig:fsm} in appendix~\ref{app:figures}.\\
\\
No assignments are done here as it is not needed.

\subsubsection{add1}
When 1 kr has been inserted into the machine, this state is reached. It counts up the TCC1 register by enabling it and setting the multiplexer to the corresponding value. This is illustrated in figure~\ref{fig:datapath} in appendix~\ref{app:figures}. On the next clock we return to the purchase state.\\
\\
One problem with this is that we have to wait for the machine to return to the purchase state to be able to receive more coin inputs. This will not pose a problem if we have a relatively fast clock, but in our case, with the manual clock, it does. It ``throws away'' coin inputs done in the add1 or add2 state. This problem is discussed in section~\ref{sec:implementation_issues}

\subsubsection{add2}
This is similar to the kr1 state. It just adds to the TCC2 register instead.

\subsubsection{return\_all}
This state simply clears the TCC1 and TCC2 registers (symbolizing a coin return), sets the returned\_all\_coins register high and goes the wait\_for\_user state.

\subsubsection{return\_change}
This state sets the returned\_changed register, decrements CC1 by one and goes to the dispense state.

\subsubsection{merge\_counter}
This is a state that was added relatively late in the process. Our original design was to merge the coin counters in the wait\_for\_user state, but this did not work. This was caused by the fact that we cycled the wait\_for\_user state while the user did not press then purchase\_finished button. As a result, our counters were merged on every clock rising edge. This was no problem with a manual clock, as you could just not clock, but with a clock of just 5Hz the counters would be merged 5 times every second.\\
\\
The solution was to go to a merge state that adds TCC1 with CC1 and TCC2 with CC2 before dispensing. We still have our amount\_eq\_cost and amount\_gt\_cost to guide ud to either the dispense or  the return\_change state.

\subsubsection{dispense}
This state releases the can and closes the coin slot. These are both D-flip-flops

\subsubsection{wait\_for\_user}
Here we just wait for the user to take the item. Otherwise the machine would go back to init, and purchase too fast for the user too see on a fast clock.\\
This state goes to init, and the process can start again.

\subsection{Datapath}
\subsubsection{Cutting the fat}
In a early implementation we had a amount register. This was trivial and involved a lot of housekeeping on it. When designing the datapath we realized that we could do without, as the only things we were really interested in could be calculated on the fly using our TCC1 and TCC2 registers. By left shifting the TCC2 register and adding it with the TCC1 register we now had a bus containing the current amount. Getting the amount\_gt\_cost and amount\_eq\_cost was now just a matter of doing simple comparisons on the bus and the cost register.


\subsection{Implementation issues}
\label{sec:implementation_issues}
\subsubsection{The missing coins}
\label{sec:missing_coins}

\paragraph{Req.1 One product, a coke-can.}
The cocke-can is the only product our vending machine can offer. This make the purchase senario of one coke-can our main senario. hence all our code revolves around this very requirement.

\paragraph{Req.2 When a sufficient amount is inserted, one cocke-can will be dispensed from the machine.} From the Purchase-state we will change to the Dispence-state if sufficient money has been inserted, that will lead to the dispence of one coke-can.

\begin{lstlisting}[caption={[VHDL]eks. text }]

\end{lstlisting}
  
\paragraph{Req.3 The product is fixed to 7 kr. ( in design this should be changeble ).} By changing this signal the price of one coke-can can can be changed. Unfortunately there are no inputs available for this action. we have agreed that a recompile would be neede to change the value. 
\begin{lstlisting}[caption={[VHDL]eks. text }]

\end{lstlisting}


\paragraph{Req.4 System accepts 1 and 2 Kr coins.}
lalala
\begin{lstlisting}[caption={[VHDL]eks. text }]

\end{lstlisting}
 

\paragraph{Req.5 A display should show the total amount inserted.}
lalala
\begin{lstlisting}[caption={[VHDL]eks. text }]

\end{lstlisting}
  

\paragraph{Req.6 A LED should indicate if change is avaliable (min 1 kr coin available).}
lalala
\begin{lstlisting}[caption={[VHDL]eks. text }]

\end{lstlisting}
 

\paragraph{Req.7 Remove product sensor, will close the the coin slot and turn on indicating LED until product is removed.}
lalala
\begin{lstlisting}[caption={[VHDL]eks. text }]

\end{lstlisting}  


\paragraph{Req.8 Return Change.}
lalala
\begin{lstlisting}[caption={[VHDL]eks. text }]

\end{lstlisting}
 

\paragraph{Req.9 Return Coins.}
lalala
\begin{lstlisting}[caption={[VHDL]eks. text }]

\end{lstlisting}


\paragraph{Req.10 If change is not available, the machine should only accept purchases with the right amount.}
lalala
\begin{lstlisting}[caption={[VHDL]eks. text }]

\end{lstlisting}

