%% LyX 1.6.5 created this file.  For more info, see http://www.lyx.org/.
%% Do not edit unless you really know what you are doing.
\documentclass[10pt,english]{article}
\usepackage[T1]{fontenc}
\usepackage{listings}

\usepackage[latin9]{inputenc}
\usepackage[letterpaper]{geometry}
\geometry{verbose,tmargin=2.5cm,bmargin=2.5cm,lmargin=2.5cm,rmargin=2.5cm}
\usepackage{amsthm}
\usepackage{amsmath}

\makeatletter

%%%%%%%%%%%%%%%%%%%%%%%%%%%%%% LyX specific LaTeX commands.
%% Because html converters don't know tabularnewline
\providecommand{\tabularnewline}{\\}

%%%%%%%%%%%%%%%%%%%%%%%%%%%%%% Textclass specific LaTeX commands.
\numberwithin{equation}{section}
\numberwithin{figure}{section}

\makeatother

\usepackage{babel}

\begin{document}
\begin{center}
\textsf{\textsc{02321 Hardware/Software Programmering (E10)}}
\par\end{center}

\begin{center}
{\Large Home assignment 5}
\par\end{center}{\Large \par}

\begin{center}
Morten Hillebo (s072923)\\
Kim Rostgaard Christensen (s084283)\\
Group 2

\par\end{center}


\subsection*{Problem 7.6 }

\begin{tabular}{|c|c|}
\hline 
A & x300A\tabularnewline
\hline
B & x3005\tabularnewline
\hline 
C & x300B\tabularnewline
\hline 
D & x3001\tabularnewline
\hline 
E & x3004\tabularnewline
\hline 
F & x3007\tabularnewline
\hline
\end{tabular}\\
\\
\begin{tabular}{|c|r|c|}
\hline 
Label & Asm & Bin\tabularnewline
\hline
\hline 
D & LD R1,+8 & 0010 001 000001000\tabularnewline
\hline 
E & ADD R1,R1,\#-1 & 0001 001 001 1 11111\tabularnewline
\hline 
F & BRp -3 & 0000 001 111111101\tabularnewline
\hline
\end{tabular}


\subsection*{Problem 8.2 }

Why is a ready bit not needed if synchronous I/O is used?\\
\\
In synchronous I/O we assume that data is always available, at
the predefined monotonic rate. If the keyboard skipped a cycle, the
char from the last cycle would be read - assuming that the register
is not cleared.


\subsection*{Problem 8.6 }

What problem could occur if a program does not check the Ready bit
of the KBSR before reading the KBDR?\\
\\
The result would be that you effectively did synchronous I/O, as
you assume data is always available. The problem would be that you
would not be able to distinguish a press on 'a' and then 'a' from
one press on 'a' for example.


\subsection*{Problem 8.10 }

What problem could occur if the display hardware does not check the
DSR before writing to the DDR?\\
\\
The DSR indicates that the display is done updating from the DDR.
If you write to DDR while it is being read by the display, the data
could be corrupted.

\newpage
\subsection*{Memory mapped IO with polling}
\lstinputlisting[language={[x86masm]Assembler},numbers=left]{Assignment_6_2.asm}
\end{document}
