\documentclass[presentation]{beamer}   % to compile the presentation
% \documentclass[handout]{beamer}        % to compile 2x2 handouts
\usepackage[ansinew]{inputenc}
\usepackage[T1]{fontenc}
\usepackage{lmodern,textcomp}
\usepackage{breakurl}

\usepackage{listings}
\usetheme{dtu}

\begin{document}

% The DTU and MIC logos
\pgfdeclareimage[height=2cm]{dtulogo}{dtu_logo}
\pgfdeclareimage[width=0.8\framesep]{miclogo}{mic_logo}

\author{Kim Rostgaard Christensen}
\title{SPARK - Correctness by Construction}

\date{Nov 30 2010}
\institute[%
  Dept. of Informatics and Mathematical Modelling
  ]{%
  DTU -- Technical University of Denmark \\
}
% \titlegraphic{\pgfuseimage{dtulogo}} % Graphics for title slide
%\logo{\pgfuseimage{miclogo}} % The left logo

\begin{frame}
  \maketitle
\end{frame}

\begin{frame}
  \frametitle{Outline}
  \tableofcontents
\end{frame}

\section{Correctness by construction}
\begin{frame}
  \frametitle{Correctness by construction}
  \framesubtitle{Putting engineering back into software engineering}
  \begin{itemize}
    \item CbyC challenges testing and debugging
    \item Manifested by the SPARK toolkit
    \item Proves the correctness of software
    \item  - By subsetting the Ada programming language
  \end{itemize}
\end{frame}

\begin{frame}
  \frametitle{Subsetting the language}
  \framesubtitle{Static programs for static analysis}
  \begin{itemize}
    \item No dynamic behaviour
      \begin{itemize}
        \item Only bounded types
        \item No pointers
        \item No aliasing
        \item No tasking/threading (will be relaxed)
      \end{itemize}
    \item All sizes known at compile-time
  \end{itemize}
\end{frame}

\begin{frame}
  \frametitle{Is subsetting enough?}
  \framesubtitle{Warning: trick question}
      \begin{itemize}
        \item Introduce a formalism
        \item Elaborates requirements 
        \item Design-by-contract
      \end{itemize}
\end{frame}

\section{Design-by-contract}
\begin{frame}
  \frametitle{Design-by-contract?}
  \framesubtitle{The fine print}
  \begin{itemize}
    \item Translate requirements to a formal language
    \item Use this as the specification
    \item Hold the implementation to this
	\item Use automated tools
      \begin{itemize}
        \item Unambiguous
        \item Not subject to human error
        \item Human code review -> producing proof
      \end{itemize}
%    \item Upon validation failure, either
%      \begin{itemize}
%        \item The implementation is wrong (incomplete)
%        \item The specification is wrong; redesign!
%      \end{itemize}  
  \end{itemize}
\end{frame}

\section{Example}
\begin{frame}
  \frametitle{Example contract}
  \framesubtitle{How does it work}
 
  \begin{block}{Quick question: What does this procedure do?}
 \lstinputlisting[language=Ada]{ada_example.ada}
 \end{block}
\end{frame}
\begin{frame}
  \frametitle{Example contract}
  \framesubtitle{How does it work}
  \begin{block}{Is this implementation correct ?}
 \lstinputlisting[language=Ada]{ada_example_impl.ada}
 \end{block}
\end{frame}

\begin{frame}
  \frametitle{Example contract}
  \framesubtitle{Elaboration}
  \begin{block}{Annotated correctly}
 \lstinputlisting[language=Ada]{spark_example.ada}
 \end{block}

\end{frame}

\section{Proving correctness}
\begin{frame}
  \frametitle{Proving correctness by static analysis}
  \framesubtitle{Quod erat demonstrandum}
   \begin{itemize}
        \item Uses pre- and postconditions
        \item Data flow analysis
        \begin{itemize}
         \item Starts by the postconditions
         \item  - and seeks to deduct the preconditions
         \end{itemize}
        \item Converts postconditions to conclusions
        \item Constructs hypothesises by visiting all branches
         \begin{itemize}
         \item The rest is simple reduction
         \item (don't worry, there is a tool)
         \end{itemize}
        \item Internally done by matrix calculations
      \end{itemize}  
\end{frame}

\begin{frame}
  \frametitle{Outlook}
  \framesubtitle{Development phases}
 \begin{enumerate}
   \item Convert informal requirements to specification
   \item Use a lot of time on design
   \item Validate specification
   \item Break down implementation modules
   \item Write one module at a time, proving it after
   \item Test and debug
  \end{enumerate}
\end{frame}

\begin{frame}
  \frametitle{Gains}
  \framesubtitle{Warning: limited information}
  Productivity is lines of code (loc) per day, and defects are measured per kloc.
\begin{table}[h]
\centering
    \begin{tabular}{ | l | l | r | r | r |}
    \hline
    Project & Year & Size & Productivity & Defects \\ \hline 
    \hline
    CDIS      & 1992 & 197 kloc & 12.7 & 0.75 \\ \hline
    SHOLIS    & 1997 &  27 kloc &  7.0 & 0.22 \\ \hline
    MULTOS CA & 1999 & 100 kloc & 28.0 & 0.04 \\ \hline
    A         & 2001 &  39 kloc & 11.0 & 0.05 \\ \hline
    NSA       & 2003 &  10 kloc & 38.0 & 0 \\ \hline        
    \end{tabular}
\caption{Comparison of projects using SPARK}
\label{table:spark_projects_comparison}
\end{table}      
\end{frame}

\section{Gains}
\begin{frame}
  \frametitle{Gains}
  \framesubtitle{The doubleplusgood}
  \begin{itemize}
        \item Formal and systematic approach to development
        \item Produce low defect rate on software
        \begin{itemize}
          \item Increase reliability and availability
        \end{itemize}
        \item Proof == 100\% correctness (according to specification)
        \item Forces you to spend time on design
        \item Provokes a lot of reflection (provability)
        \item Encourages modularity -> improves maintainability
        \item Guarantees analysis done in P time
      \end{itemize}       
\end{frame}

\section{Challenges}
\begin{frame}
  \frametitle{Challenges}
  \framesubtitle{The doubleplusungood}
  \begin{itemize}
        \item Fairly uncharted waters - no wide adoption
        \item Steep learning curve, not for the average programmer
        \item Strong bounds on language features
        \item Paradigm shift
      \end{itemize}       
\end{frame}

\section{Final words}
\begin{frame}
  \frametitle{Real-time domain language}
  \framesubtitle{Ada terminology}
  \begin{itemize}
     \item Run-time has built-in dispatcher
     \item Ada task == Real-time task
     \item Compiler supports locking policies
   \end{itemize}       
\end{frame}

\begin{frame}
  \frametitle{Real-time domain language}
  \framesubtitle{Safety-critical}
  \begin{itemize}
     \item Uses profiles to assert conformance
     \item RavenSPARK profile enables SPARK tasking
     \item Schedulabiliy analysis is directly applicable
     \item Check out the Assert project
   \end{itemize}       
\end{frame}



\begin{frame}
  \frametitle{Questions?}
  \framesubtitle{}
      
\end{frame}


\end{document} 