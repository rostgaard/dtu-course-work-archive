\documentclass[10pt]{scrartcl}
\usepackage{geometry}
\usepackage[parfill]{parskip}
\usepackage{graphicx}
\usepackage{amssymb}
\usepackage{epstopdf}
\usepackage{color}
\usepackage{listings}
\lstset{
  frame=single,language=bash,
  morekeywords={od, in, foreach, let, end, and, or, proc, while},
   basicstyle=\footnotesize,
  escapechar=\@,
  basicstyle=\footnotesize, frame=tb,
  numbers=left,
  stepnumber=2,
  numbersep=5pt, 
  numberstyle=\tiny\color{mygray},
  xleftmargin=.2\textwidth, xrightmargin=.2\textwidth
}
\usepackage{geometry}
 \geometry{
 a4paper,
 total={210mm,297mm},
 left=20mm,
 right=20mm,
 top=20mm,
 bottom=20mm,
 }
\definecolor{bluekeywords}{rgb}{0.13,0.13,1}
\definecolor{greencomments}{rgb}{0,0.5,0}
\definecolor{redstrings}{rgb}{0.9,0,0}
\definecolor{mygray}{rgb}{0.2,0.2,0.2}

\DeclareGraphicsRule{.tif}{png}{.png}{`convert #1 `dirname #1`/`basename #1 .tif`.png}

\title{02257 - Applied functional programming}
\subtitle{Project1 - Interpreter}
\author{Anna Maria Walach - \textit {s121540@student.dtu.dk} \\ Kim Rostgaard Christensen - \textit {s084283@student.dtu.dk}}
\begin{document}
\maketitle
\section{Object types}
There are three main categories of objects defined in the language:
\begin{itemize}
\item procedures, represented by tuple of arguments' names, procedure environment and set of procedure statements
\item arrays, represented by tuple of length and values stored in F\# array
\item primitive values: int, string, boolean
Complex structures (procedure, array) can never be send as a value to the function - always as reference. 
\end{itemize}
\subsection{Array handling}
Array variable points to reference, that store the whole array object. Values can be assigned to array cells by calling \texttt{array[index] = newValue}. Each array has attribute "length" that can be access by \texttt{array.length}.
\section{Recursive and non-recursive procedures}
We decided to not explicitly implement the recursive procedures in our interpreter. Instead, each procedure (regardless the \texttt{rec} keyword) can include recursive calls. \\
Procedure need to be always structured with let... in ... end, unless they contain only one statement (including \texttt{return}).
\subsection{Return statements}
Procedure can, but does not have to, return a value or reference. Reference to local value can be returned. Another arrays and procedures are always returned as references, not objects. 
\subsection{Parameter passing}
The language support two ways of passing the primitive (string, int, bool) parameters to function: by reference and by value. 
\begin{lstlisting}
let x : 5; y : 10;
	proc sum (a, b)
		a := +(!a,!b)
in
	call sum(x,y);
	call sum(!x,!y)
end
\end{lstlisting}
In the first case, described by \texttt{sum(x,y)}, we're sending the references to the function, so after this line, \texttt{x := 5}. After calling \texttt{sum(!x, !y)}, \texttt{x} value remains unchanged, as only the content of values is send to the procedure.
\section{Control statements}
\subsection{Conditionals}
Conditionals are implemented, but only for boolean expressions on integer types. The if else statement is structured in following way:
\begin{lstlisting}
if (condition) then
	// do this
else //optional
	// do something else
fi
\end{lstlisting}
\subsection{Loops}
\subsubsection{While loop}
While loop is supported in a form:
\begin{lstlisting}
while (condition) do
  // do something
od
\end{lstlisting}	
\subsection{For loop - extension}
The for loop was implemented in a form:
\begin{lstlisting}
for (assignment; condition; incrementation) do
  // do something
od
\end{lstlisting}	
\texttt{Assignment} is a \emph{statement} that assigns starting value to iterating variable. \\
\texttt{Condition} is an \emph{expression} returning true or false. \\
\texttt{Incrementation} is a \emph{statement} that is responsible for incrementing the iterating variable and is called at the end of each iteration.
\subsubsection{Foreach loop - extension}
We've implemented a basic foreach language construct that, within the interpreter, translates into a while loop. An example f a foreach construction can be seen below.
  \begin{lstlisting}
foreach var in collection do
  // statements where var is accessible
od
\end{lstlisting}
\section{Infix logic operators - extension}
Initially we wanted to extend the language with infix versions the implemented operators. However, to avoid breaking already implemented test cases, we instead added two new operators -- the logic AND and OR. This enables the following syntax;
\begin{lstlisting}
let
  boolvar :  <>(x,5) or <(y,10) and >(y,0)
in
  ...
end
\end{lstlisting}
Parsing and evaluation works, but unfortunately we have a shift-reduce conflict in both logic expression that we have not been able to resolve.
\section{Test cases}
We performed 7 tests that were already included in the project and 7 we created later to test extensions:
\begin{itemize}
\item \texttt{AndOrInfix} - for testing the boolean infix operators
\item \texttt{ArrayFolding} - for testing arrays as function arguments, arrays attributes and accessing arrays cells.
\item \texttt{ArrayTestStringIndex} - test that proves the string indices are not supported
\item \texttt{IfElseParse} - for testing the ITE structure
\item \texttt{ArgsReferencesValues} - for testing the passing arguments (by value and by reference)
\item \texttt{ForeachLoop} - testing the correctness of foreach loop implementation
\item \texttt{ForLoop} - testing the correctness of for loop implementation
\end{itemize}
All tests are passing without problems and delivering proper values. 
The default tests were modified to include the assertions (if statements).

\section{Conclusions}
We were able to implement all basic concepts (procedures, while loops, ITE, arrays) and few additional (for, foreach, infix boolean operators), which we evaluate as satisfying result. \\
We can see some optimization possibilities, like rewriting the Exp and Value types, which may allow better reflection of natural code structures and properties (e.g. remove the possibility of storing non-Reference value in environment). \\
During development, we discovered that actually we can divide objects into two categories: primitives and complex (similar to Java), where primitives can be send by reference/value, and complex only be reference. Even procedures falls into the categories of complex structure, which made really easy implementing functions as procedure arguments. 
\end{document}  