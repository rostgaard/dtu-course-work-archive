\documentclass[11pt]{scrartcl}
\usepackage{geometry}
\usepackage[parfill]{parskip}
\usepackage{graphicx}
\usepackage{amssymb}
\usepackage{epstopdf}
\usepackage{color}
\usepackage{listings}
\lstset{
  frame=single,language=bash,
  morekeywords={od, in, foreach},
   basicstyle=\footnotesize,
  escapechar=\@,
  basicstyle=\footnotesize, frame=tb,
  numbers=left,
  stepnumber=2,
  numbersep=5pt, 
  numberstyle=\tiny\color{mygray},
  xleftmargin=.2\textwidth, xrightmargin=.2\textwidth
}
\definecolor{bluekeywords}{rgb}{0.13,0.13,1}
\definecolor{greencomments}{rgb}{0,0.5,0}
\definecolor{redstrings}{rgb}{0.9,0,0}
\definecolor{mygray}{rgb}{0.2,0.2,0.2}

\DeclareGraphicsRule{.tif}{png}{.png}{`convert #1 `dirname #1`/`basename #1 .tif`.png}

\title{02257 - Applied functional programming}
\subtitle{Project1 - Interpreter}
\author{Anna Maria Walach - \textit {s121540@student.dtu.dk} \\ Kim Rostgaard Christensen - \textit {s084283@student.dtu.dk}}
\begin{document}
\maketitle
\section{Non-recursive procedures}

\section{Recursive procedures}
\subsection{Return statements}
\subsection{Conditionals}
Conditionals are implemented, but only for boolean expressions on integer types.


\section{Array handling}

\section{Syntactic sugar - foreach}
We've implemented a basic foreach language construct that, within the interpreter, translates into a while loop. An example of a foreach construction can be seen below.
  \begin{lstlisting}
foreach var in collection do
  // statements where var is accessible
od
 \end{lstlisting}
 As of this writing, the extensions' grammar, AST and translation is functional, but faces three obstacles before being fully working;
 \begin{itemize}
   \item \texttt{var} is not aliased to \texttt{collection[i]}, where \texttt{i} is the i'th element in the collection. This prevents programs from actually accessing \texttt{var} within the block of the foreach loop.
   \item To make the iteration progress, an increment statement should be appended to the body of the foreach loop
   \item The \texttt{.length} attribute, as of now, only handles declared arrays and not arrays returned from expressions. As foreach takes in an expression, further extensions are needed to the attribute system. 
 \end{itemize}
   \section{Test cases}



\end{document}  