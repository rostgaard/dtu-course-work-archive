\documentclass[10pt]{scrartcl}
\usepackage{geometry}
\usepackage[parfill]{parskip}
\usepackage{graphicx}
\usepackage{amssymb}
\usepackage{epstopdf}
\usepackage{color}
\usepackage{listings}
\lstset{
  frame=single,language=bash,
  morekeywords={od, in, foreach, let, end, and, or, proc, while},
   basicstyle=\footnotesize,
  escapechar=\@,
  basicstyle=\footnotesize, frame=tb,
  numbers=left,
  stepnumber=2,
  numbersep=5pt, 
  numberstyle=\tiny\color{mygray},
  xleftmargin=.2\textwidth, xrightmargin=.2\textwidth
}
\usepackage{geometry}
 \geometry{
 a4paper,
 total={210mm,297mm},
 left=20mm,
 right=20mm,
 top=20mm,
 bottom=20mm,
 }
\definecolor{bluekeywords}{rgb}{0.13,0.13,1}
\definecolor{greencomments}{rgb}{0,0.5,0}
\definecolor{redstrings}{rgb}{0.9,0,0}
\definecolor{mygray}{rgb}{0.2,0.2,0.2}

\DeclareGraphicsRule{.tif}{png}{.png}{`convert #1 `dirname #1`/`basename #1 .tif`.png}

\title{02257 - Applied functional programming}
\subtitle{Project 2 - Drawing trees}
\author{Anna Maria Walach - \textit {s121540@student.dtu.dk} \\ Kim Rostgaard Christensen - \textit {s084283@student.dtu.dk}}
\begin{document}
\maketitle
\section{Translating solution to F\#}
\section{Generating PostScript file}
Generation of PostScript content, is a matter of mapping nodes to PostScript commands. In our case we have found that the following macro works well.
\begin{lstlisting}[language=PostScript]
$node_x $node_y            moveto
($label)                   dup stringwidth pop 2 div neg 0 rmoveto show
$node_x $padded_node_y     moveto
$parent_x $parent_y        lineto stroke
\end{lstlisting}
the \$ denotes values generated in the F\# program. This block adds a centered label and draws a direct line from a child node to its parent. The handed in version of the program draws angled lines instead of direct ones.
We generate the drawing top-down, as the algorithm from \cite{kennedy1996functional} already handled the tree fitting for us, and pick an absolute point on the PS canvas and start drawing there.
PostScript files contain both footers and headers that needs to be prepended and appended to the PostScript content, respectively. This is done by declaring these as string constants which are concatenated with generated content in the middle.
\section{Visualizing AST's}
We visualize our abstract syntax trees from Project 1 by recursing trough them and converting each AST node into a Tree Node. We've made two different implementations; one that maps every node within the AST to a Tree node, and one that tries to make it a bit more condense and readable tree. 
\subsection{Basic version}
In the basic version we treat Blocks as nodes having two children; declarations and statement body. Statement lists are nodes with children who are their contained statements. Function applications have two child nodes; arguments and their statement body. The remaining statement, expressions and declarations are mapped more or less directly, or uses the rules from Block or StmList already covered.


the F\# function \texttt{st, ex, dec} makes large tree, and the \texttt{stToTree, } equivalents makes the compact versions.
\section{Efficiency}

Typical concatenation
Real: 00:00:00.001, CPU: 00:00:00.000, GC gen0: 0, gen1: 0, gen2: 0\\
> Real: 00:00:00.008, CPU: 00:00:00.000, GC gen0: 1, gen1: 0, gen2: 0\\
Real: 00:00:00.009, CPU: 00:00:00.015, GC gen0: 2, gen1: 1, gen2: 1\\
> Real: 00:00:00.011, CPU: 00:00:00.015, GC gen0: 0, gen1: 0, gen2: 0\\
Real: 00:00:00.012, CPU: 00:00:00.015, GC gen0: 0, gen1: 0, gen2: 0\\
> Real: 00:00:00.009, CPU: 00:00:00.015, GC gen0: 1, gen1: 0, gen2: 0\\\\
StringBuilder concatenation
Real: 00:00:00.001, CPU: 00:00:00.000, GC gen0: 0, gen1: 0, gen2: 0\\
> Real: 00:00:00.011, CPU: 00:00:00.000, GC gen0: 0, gen1: 0, gen2: 0\\
Real: 00:00:00.005, CPU: 00:00:00.000, GC gen0: 0, gen1: 0, gen2: 0\\
> Real: 00:00:00.007, CPU: 00:00:00.015, GC gen0: 1, gen1: 0, gen2: 0\\
Real: 00:00:00.005, CPU: 00:00:00.000, GC gen0: 0, gen1: 0, gen2: 0\\
Real: 00:00:00.007, CPU: 00:00:00.015, GC gen0: 0, gen1: 0, gen2: 0\\\\
> String.concat concatenation
Real: 00:00:00.001, CPU: 00:00:00.000, GC gen0: 0, gen1: 0, gen2: 0\\
Real: 00:00:00.012, CPU: 00:00:00.000, GC gen0: 0, gen1: 0, gen2: 0\\
> Real: 00:00:00.005, CPU: 00:00:00.000, GC gen0: 0, gen1: 0, gen2: 0\\
Real: 00:00:00.007, CPU: 00:00:00.000, GC gen0: 1, gen1: 0, gen2: 0\\
> Real: 00:00:00.007, CPU: 00:00:00.000, GC gen0: 1, gen1: 0, gen2: 0\\
Real: 00:00:00.010, CPU: 00:00:00.000, GC gen0: 2, gen1: 0, gen2: 0\\
\section{Extensions}
\section{Conclusions}
We have not tested the limit on how big trees we can generate, but are clearly limited by not using tail recursion in our drawing algorithm. A nice improvement would be to make the algorithm tail recursive.

\bibliographystyle{plain}
\bibliography{references}

\end{document}  
