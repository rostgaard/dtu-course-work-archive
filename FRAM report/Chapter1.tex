% notes/todo

TODO: Something about the dispatchers and about what happen when things go right (near-miss)


Traditionally, safety systems are modelled as complex linear system, rather than non-linear.

\chapter{Terminology}
In general it is not recommended to say that a specific event (X) causes another (Y). This implies that X is a precondition to Y, and by eliminating X\footnote{Methods for accident investigation}

(stolen)
The term Resilience Engineering represents a new way of thinking about safety. Whereas conventional risk management approaches are based on hindsight and emphasise error tabulation and calculation of failure probabilities, Resilience Engineering looks for ways to enhance the ability of organisations to create processes that are robust yet flexible, to monitor and revise risk models, and to use resources proactively in the face of disruptions or ongoing production and economic pressures. In Resilience Engineering failures do not stand for a breakdown or malfunctioning of normal system functions, but rather represent the converse of the adaptations necessary to cope with the real world complexity. Individuals and organisations must always adjust their performance to the current conditions; and because resources and time are finite it is inevitable that such adjustments are approximate. Success has been ascribed to the ability of groups, individuals, and organisations to anticipate the changing shape of risk before damage occurs; failure is simply the temporary or permanent absence of that.
(end stolen)

Safety: freedom from unacceptable risk

TODO: explain risk matrix

\section{Resonance}

Resonance is a phenomenon in physics making a system oscillate at a higher amplitude when a force is applied.

\section{ATC}
ATC, or Automatic Train Control

Barrier in safety engineering

general safety en resilience engineering terminology

\section{Railway}

\section{Safety systems}

\section{Classic model on safety}
%http://en.wikipedia.org/wiki/Swiss_cheese_model
Barriars are depicted as layers of swiss cheese with holes in them. 

\section{Resilience Engineering}
\label{sec:resilience_engineering}

\section{Human reliabilty assessment models}



\subsection{Functional Resonance Analytic Model}

Difference between method and model.