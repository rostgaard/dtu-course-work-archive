\chapter{Example accident model}
\label{ch:accident_model}
Using FRAM restrospectively will, hopefully, identify some of the critically constrained couplings between functions in a system. Appling this knowlegde, it will be possible not only to build safer, but also more resillient system.

\section{Near miss at Train crossing}
Modelling a near miss incident is not very common, though very rewarding in terms of drawing experience from them. A more elaborate explanation of what a near miss is, see section \ref{sec:near_miss}

In this thesis, we will model a simple near miss incident.

%Previous studies\cite{belmonte2011interdisciplinary}

At Grenåbanen tuesday the 26th of March 2010, at 14:40, an ambulance was intentionally led over railway crossing that should have been secured. This situation led to a near-miss, and luckily no one was harmed.\\

\subsection{Background}
From the accident report:
Train RV 4940 in transit from Grenå towards Aarhus was signalled that crossing 128a was secured.

Shortly after, the train driver realized that 2 railway employees were located on the track. The driver did not do anything further as he assumed they would move when they saw the train.

As the train approached the crossing, an ambulance with siren signal entered the crossing - from the road side. The train driver used the emergency brake, hereby avoiding collision with the ambulance. According the the train driver, the collision was imminent.\\
\\
The 2 railway employees reported that, they though they would be able to assist the ambulance in reaching its destination faster, by leading it into the crossing before the train arrived, by misjudged the situation.\\
\\
The document ``udrykningsbekendtgørelsen'' (the official notice regarding emergency) states that the driver of an emergency vehicle, must at all times abide signals or other instructions, at railway crossings

Following the initial investigations and evaluations of the data available - the accident investigation committee reached the following conclusion; further studies would not necessarily lead to preventative recommendations, or result in findings leading to significant improvements in railway safety.

With reference to Danish railway legislation, the accident investigation committee decided not to perform further studies. 

\subsubsection{FRAM model}
TODO
\section{Discussion}
%TODO In general, like in statistics, Corrolation does not imply causation.
