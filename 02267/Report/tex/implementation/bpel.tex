\section{BPEL}
This section describes how we have implemented TravelGood as a BPEL process. We will go over the data types we have used, how we designed the process as well as general design decisions we made.

\subsection{The Port}
\pet

\noindent
Since all the services work together on the same process, we have only made one port for TravelGood. We use RPC-literal style for this, with request/response operations. As a client for the service, would need to update the user on the results of all operations, we need to wait for some sort of response for every operation.

\subsection{Business Process}
\mrb

\noindent
We have split the business process into two phases, which corresponds to some of the states shown in Fig.~\ref{fig:protocol}. These phases are:

\begin{description}
\item [Planning:] In this phase, the user is planning his travel and can freely add hotels and flights he find using the appropriate search methods. He can, at any point choose to cancel the itinerary. Once he is done, he can book the itinerary to move to the next phase.

\item[Booked:] If the booking of all hotels and flights was successful, the process moves to this phase. The user can now cancel the bookings, but it might not work for all bookings. If the cancellation was successful, the process is passive until one day before the first day of travel. If the cancellation fails for one or more of the items on the itinerary, we stay in this state, where the status of items that could be cancelled is updated. We always leave this phase and end the process at latest, one day before the first item on the itinerary.

\end{description}

\subsubsection{Planning}
\pet

\noindent
During this phase, BPEL accepts the following messages

\begin{description}
\item [\texttt{getFlights}:] BPEL forwards the relevant information to LameDuck and returns the result to the client.
\item [\texttt{getHotels}:] BPEL forwards the relevant information to NiceView and returns the result to the client.
\item [\texttt{addFlight}:] BPEL adds the flight to the planned flights in the itinerary, LameDuck is not contacted in this step.
\item [\texttt{addHotel}:] BPEL adds the hotel to the planned hotels in the itinerary, NiceView is not contacted in step.
\item [\texttt{getItinerary}:] BPEL returns the lists of hotels and flights along with the current status of them (``\texttt{unconfirmed}'').
\item [\texttt{cancelPlanning}:] BPEL ends the process which in turn drops the current planned bookings.
\item [\texttt{bookItinerary}:] BPEL attempts to book all the planned flights and hotels. If successful, we move on the next phase. If any items fails to book, we compensate the ones that have been booked and stay in the planning phase. The client is informed of the result.
\end{description}

We have implemented these as a \texttt{pick} activity, inside a loop that runs as long as a boolean variable indicates that we are in the planning phase. The actual booking happens inside a scope which first loops over each flight, and then each hotel and books them if possible. It has been done sequentially with flights first and then hotels, such that we can reliable make a certain booking fail in order to comply with a few of the mandatory unit tests. 

A compensation handler is attached on the scope of each flight and hotel booking, such that it can be compensated if necessary. On the scope where the planning \texttt{pick} activity resides, we have a fault handler that will call the corresponding compensation handlers when a booking fails.

\subsubsection{Booked}
\mkt

\noindent
During this phase, BPEL accepts the following messages.

\begin{description}
\item [\texttt{getItinerary}:] BPEL returns the lists of hotels and flights along with the current status of them.
\item [\texttt{cancelItinerary}:] BPEL will cancel as many items as possible. The client is informed if any of the cancellations failed by returning false, and the process will remain active until one day before the travel was to start. If it was successful the process will terminate.
\end{description}

This phase is also implemented as a \texttt{pick} activity. If we get a \texttt{cancelItinerary} request, we attempt to cancel all booked items. Any that fails, remain booked, while the others get the status ``\texttt{cancelled}''. If any cancellations fails the itinerary information will remain available upon request until one day before the travel was meant to begin. 

The other possibility of the pick is that a timeout happens. This timeout models that it is one day before the first item on the itinerary and the whole thing is removed and the process is ended.


\subsection{Implementation details}
\kim

\noindent
In this implementation we have simplified the part of "wait until one day before travel begins before ending process", since this would make it harder to unit test it. In our case we have just implemented a wait of 10 seconds before it ends the process. If we wanted to implement the proper wait, we would use the XPath expression \texttt{hours-from-dateTime} to get the time the itinerary should be terminated and set the wait activity to fire at this time.
