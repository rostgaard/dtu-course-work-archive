%%%%%%%%%%%%%%%%%%%%%%%%%
% LaTeX style redefined %
%%%%%%%%%%%%%%%%%%%%%%%%%

%%%%%%%%%%%%
% Packages %
%%%%%%%%%%%%
\usepackage[utf8]{inputenc}
\usepackage[british]{babel}
\usepackage{amsmath,amssymb,latexsym} 	% AMS and other symbols 
\usepackage{float}                    	% Improves the interface for defining floating objects
\usepackage{fancyhdr}                 	% Pretty headers/footers
\usepackage{hyperref}                 	% References package – hyperref load after fancyhdr
\usepackage{etoolbox}                 	% Patching commands
\usepackage{graphicx}                 	% Support for including graphics
\usepackage{xcolor}                   	% Support for using colours
%\usepackage{parskip}                  	% Disable indentation and make a newline after pargrahps
\usepackage[noabbrev,nameinlink]{cleveref}   % Clever references. Options: "fig. [1]" --> "[Figure 1]"
\usepackage{tikz}                     	% Drawing tool
\usepackage{forloop}                  	% Enabling for-loops
\usepackage{sectsty}                  	% Style redefinition on multiple depth of sections
\usepackage{titlesec}                 	% Chapter style
\usepackage{import}                   	% Establish input relative to a directory
\usepackage{listings}                 	% Source code printer for LaTeX
\usepackage{avant}                    	% Sans-serif font. Try also `helvet'
\usepackage{lipsum}						% Lipsum for random latin text paragraphs
\usepackage{epstopdf}					% .eps to .pdf package
\usepackage[a4paper]{geometry} 			% Slightly wider margins than default
\usepackage{caption}					% Expanded captions
\usepackage{subcaption}					% Subcaptions
\usepackage{color}						% Color package
\usepackage[normalem]{ulem}				% 


%%%%%%%%%%%%%%%%%%%%%%%%%%%%%%%%%%%%
% Setup (only one liners)          %
%%%%%%%%%%%%%%%%%%%%%%%%%%%%%%%%%%%%
\selectlanguage{british}
\usetikzlibrary{arrows,shapes,positioning,topaths}
\graphicspath{{\confdir img/}{\imgdir}}
\newcounter{cnt}
\newcommand{\eval}[1]{\the\numexpr #1\relax}
\newcommand{\insertcode}[2]{\subsubsection*{#2}\lstinputlisting[style=Java]{\srcdir #1/#2}}

%%%%%%%%%%%%%%%%%
% Uncategorized %
%%%%%%%%%%%%%%%%%

% Make verbatim use \scriptsize
\makeatletter
\g@addto@macro\@verbatim\scriptsize
\makeatother

%%%%%%%%%%
% Colors %
%%%%%%%%%%
% See https://www.dtu.dk/upload/dtu%20kommunikation/designguide/designplatform_farver2010.pdf
\definecolor{dtured}{cmyk}{0,.91,.72,.23}
\definecolor{dtugray}{cmyk}{0,0,0,.56}
%\definecolor{s01}{cmyk}{}
%\definecolor{s02}{cmyk}{}
%\definecolor{s03}{cmyk}{}
%\definecolor{s04}{cmyk}{}
\definecolor{s05}{cmyk}{.0,1,.0,.0}    % magenta
\definecolor{s06}{cmyk}{.25,1,.0,.0}
\definecolor{s07}{cmyk}{.25,1,0,0}
\definecolor{s08}{cmyk}{.75,1,0,0}
\definecolor{s09}{cmyk}{.75,.75,.0,.0}
\definecolor{s10}{cmyk}{.5,0,0,0}
\definecolor{s11}{cmyk}{.5,.0,.0,.0}
%\definecolor{s12}{cmyk}{}
\definecolor{s13}{cmyk}{.75,.50,.0,.0} % blue
\definecolor{s14}{cmyk}{.5,0,1,0}      % green
\definecolor{s14a}{cmyk}{.6,0,1,.25}      % green

%%%%%%%%%%%%%%%
% Hyper setup %
%%%%%%%%%%%%%%%
\hypersetup{
    bookmarksnumbered=true,
    bookmarksopen,
    breaklinks,
    colorlinks,
    linktoc=all,
    plainpages=false,
    unicode=true,
    citecolor=dtured,          % color of links to bibliography
    filecolor=dtured,          % color of file links
    linkcolor=blue,          % color of internal links (change box color with linkbordercolor)
    urlcolor=s13         % color of external links
}

%%%%%%%%%%%%%
% Titlepage %
%%%%%%%%%%%%%
\newcommand{\HRule}{\rule{\linewidth}{0.5mm}}
\newcommand{\inserttitlepage}{\pagenumbering{alph}\input{\confdir titlepage}}

%%%%%%%%%%%%%%%%%%%%%
% Table of contents %
%%%%%%%%%%%%%%%%%%%%%
\newcommand{\inserttoc}{\pagenumbering{roman}\tableofcontents\newpage\pagenumbering{arabic}\setcounter{page}{1}}

\setcounter{tocdepth}{2}
\setcounter{secnumdepth}{2}

%%%%%%%%%%%%%%%%%
% Chapter style %
%%%%%%%%%%%%%%%%%
\newlength{\chapterfontsize}
\setlength{\chapterfontsize}{28pt}
\titleformat{\chapter}[display]
    {\sffamily}                            % format
{\large\textsc{\color{dtugray}\chaptertitlename}~~\fontsize{34}{34}\selectfont\color{dtured}\thechapter} % label
    {-3ex}                                 % seperation between label and chapter title
    {\filleft\fontsize{\chapterfontsize}{\chapterfontsize}\selectfont} % before chapter title
    [\hrule]                               % after chapter title

\titlespacing*{\chapter}{0pt}{0pt}{10pt}

%%%%%%%%%%%%%%%%%%%%%%%%%%%
% Header and footer setup %
%%%%%%%%%%%%%%%%%%%%%%%%%%%
\pagestyle{fancy}

\fancyhead{}
\fancyfoot{}
\renewcommand{\headrulewidth}{0.4pt}
\renewcommand{\footrulewidth}{0.4pt}
%\renewcommand{\chaptermark}[1]{\markboth{\thechapter.\ #1}{}}
\renewcommand{\sectionmark}[1]{\markboth{\thesection.\ #1}{}}

% Header
\lhead{\footnotesize\sffamily \nouppercase{\sffamily\leftmark}}
\chead{}
\rhead{}

% Footer
\lfoot{\tiny\sffamily
    \begin{tabular}{ll} 
        \forloop[2]{cnt}{1}{\value{cnt} < \eval{\value{authorcnt}+1}}{%
            \authorname{\thecnt} & \authorname{\eval{\thecnt+1}} \\
        }
    \end{tabular}
}
\cfoot{}
\rfoot{\small \thepage}

%%%%%%%%%%%%%%%%%%%%%%%%%%%%%%%%%%%%%%%%%%
% Course and subject configuration setup %
%%%%%%%%%%%%%%%%%%%%%%%%%%%%%%%%%%%%%%%%%%
\def\courseno{\textit{COURSE NUMBER}}
\def\coursename{\textit{COURSE NAME}}
\def\title{\textit{TITLE}}
\def\subtitle{\textit{SUBTITLE}}
\def\date{\textit{DATE}}

\newcommand{\setcourse}[2]{%
    \def\courseno{#1}%
    \def\coursename{#2}%
}
\newcommand{\settitle}[1]{%
    \def\title{#1}%
}
\newcommand{\setsubtitle}[1]{%
    \def\subtitle{#1}%
}
\newcommand{\setdate}[1]{%
    \def\date{#1}%
}

%%%%%%%%%%%%%%%%%%%%%%%%%%%%%%
% Author configuration setup %
%%%%%%%%%%%%%%%%%%%%%%%%%%%%%%
\newcounter{authorcnt}
% Add a new author
\newcommand{\addauthor}[3]{%
    \stepcounter{authorcnt}%
    \csdef{author_sno\theauthorcnt}{#1}
    \csdef{author_fname\theauthorcnt}{#2}
    \csdef{author_lname\theauthorcnt}{#3}
}  
% Set a specific author
\newcommand{\setauthor}[4]{%
    \csdef{author_sno#1}{#2}
    \csdef{author_fname#1}{#3}
    \csdef{author_lname#1}{#4}
}

% Get formatted name of author
\newcommand{\fauthorname}[1]{%
    \csuse{author_fname#1} \textsc{\csuse{author_lname#1}}%
}

% Get unformatted name of author
\newcommand{\authorname}[1]{%
    \csuse{author_fname#1} \csuse{author_lname#1}%
}

% Get study number of author
\newcommand{\authorsno}[1]{%
    \csuse{author_sno#1}%
}

%%%%%%%%%%%%%
% Todo list %
%%%%%%%%%%%%%
\makeatletter \newcommand \listoftodos{\section*{Todo list} \@starttoc{tdo}\newpage}
\newcommand\l@todo[2]
{\noindent \textit{#2} : \parbox{10cm}{#1} \par\vspace{8pt}} \makeatother

\newcommand{\todo}[1]{
   \addcontentsline{tdo}{todo}{\protect{#1}}
   \marginpar{#1}
}

%%%%%%%%%%%%
% Listings %
%%%%%%%%%%%%
\crefname{lstlisting}{listing}{listings}
\Crefname{lstlisting}{Listing}{Listings}
\lstset{%
  backgroundcolor=\color{white},  % choose the background color; you must add \usepackage{color} or \usepackage{xcolor}
  basicstyle=\footnotesize\ttfamily,       % the size of the fonts that are used for the code
  belowskip=-10pt plus 5pt,
  breakatwhitespace=false,        % sets if automatic breaks should only happen at whitespace
  breaklines=true,                % sets automatic line breaking
  captionpos=b,                   % sets the caption-position to bottom
  commentstyle=\color{s14a},      % comment style
  escapeinside={(*}{*)},          % if you want to add LaTeX within your code
  frame=single,                   % adds a frame around the code
  keywordstyle=\bf\ttfamily\color{s09},      % keyword style
  morekeywords={*,...},           % if you want to add more keywords to the set
  numbers=left,                   % where to put the line-numbers; possible values are (none, left, right)
  numbersep=0pt,                  % how far the line-numbers are from the code
  numberstyle=\sffamily\tiny\color{dtugray},  % the style that is used for the line-numbers
  rulecolor=\color{dtugray},      % if not set, the frame-color may be changed on line-breaks within not-black text (e.g. comments (green here))
  showspaces=false,               % show spaces everywhere adding particular underscores; it overrides 'showstringspaces'
  showstringspaces=false,         % underline spaces within strings only
  showtabs=false,                 % show tabs within strings adding particular underscores
  stepnumber=1,                   % the step between two line-numbers. If it's 1, each line will be numbered
  stringstyle=\color{s07},        % string literal style
  tabsize=2,                      % sets default tabsize to 2 spaces
  caption=\lstname                  % show the filename of files included with \lstinputlisting; also try caption instead of title
}

\lstdefinestyle{Java}{
    backgroundcolor=\color[RGB]{240,240,255},
    basicstyle=\footnotesize\ttfamily,
    commentstyle=\ttfamily\color{green!40!black},
    escapeinside={(*}{*)},
    identifierstyle=\color{black},
    keywordstyle=\bfseries\color{blue},
    language=Java,
    stringstyle=\color{orange},
    tabsize=2,
}

\lstdefinestyle{Bash}{
    basicstyle=\scriptsize\ttfamily\color{black},
    language=bash,
    numbers=none,
    numberstyle=\color{dtured}\tiny,
    literate={\$}{{\textcolor{dtured}{\$}}}1,
}



%%%%%%%%%%%%%%%%
% Packed lists %
%%%%%%%%%%%%%%%%

\newenvironment{packed_enum}{
\begin{enumerate}
  \setlength{\itemsep}{0pt}
  \setlength{\parskip}{0pt}
  \setlength{\parsep}{0pt}
}{\end{enumerate}}

\newenvironment{packed_item}{
\begin{itemize}
  \setlength{\itemsep}{0pt}
  \setlength{\itemindent}{-10pt}
  \setlength{\parskip}{0pt}
  \setlength{\parsep}{0pt}
}{
\end{itemize}
}
