\documentclass[10pt,a4paper]{article}
\usepackage[utf8]{inputenc}
\usepackage{url}
\usepackage{amsmath}
\usepackage{amsfonts}
\usepackage{amssymb}
\usepackage{listings}
\usepackage{graphicx}
\usepackage[scaled]{beramono}
%\usepackage{bera}

\usepackage{libertine}
\usepackage[T1]{fontenc}


\usepackage{color}

\def\File#1{\textsf{#1}}
\def\Code#1{\texttt{#1}}
\def\Key#1{\textsf{#1}}
\usepackage{color}
\definecolor{bluekeywords}{rgb}{0.13,0.13,1}
\definecolor{greencomments}{rgb}{0,0.5,0}
\definecolor{redstrings}{rgb}{0.9,0,0}

\definecolor{mygreen}{rgb}{0,0.6,0}
\definecolor{mygray}{rgb}{0.2,0.2,0.2}
\definecolor{mymauve}{rgb}{0.58,0,0.82}

\usepackage{listings}
%\lstset{
%}

\lstset{ %
  language=Java,
  backgroundcolor=\color{white},   % choose the background color; you must add \usepackage{color} or \usepackage{xcolor}
  showspaces=false,
  basicstyle=\footnotesize\ttfamily,  % the size of the fonts that are used for the code
  breakatwhitespace=false,         % sets if automatic breaks should only happen at whitespace
  breaklines=true,                 % sets automatic line breaking
  captionpos=b,                    % sets the caption-position to bottom
  deletekeywords={...},            % if you want to delete keywords from the given language
  escapeinside={\%*}{*)},          % if you want to add LaTeX within your code
  extendedchars=true,              % lets you use non-ASCII characters; for 8-bits encodings only, does not work with UTF-8
  frame=single,                    % adds a frame around the code
%  keywordstyle=\color{blue},       % keyword style
%  language=Octave,                 % the language of the code
  morekeywords={*,...},            % if you want to add more keywords to the set
  numbers=left,                    % where to put the line-numbers; possible values are (none, left, right)
  numbersep=5pt,                   % how far the line-numbers are from the code
  numberstyle=\tiny\color{mygray}, % the style that is used for the line-numbers
  rulecolor=\color{mygray},         % if not set, the frame-color may be changed on line-breaks within not-black text (e.g. comments (green here))
  showspaces=false,                % show spaces everywhere adding particular underscores; it overrides 'showstringspaces'
  showstringspaces=false,          % underline spaces within strings only
  showtabs=false,                  % show tabs within strings adding particular underscores
  stepnumber=2,                    % the step between two line-numbers. If it's 1, each line will be numbered
  commentstyle=\color{greencomments},
  keywordstyle=\color{bluekeywords}\bfseries,
  stringstyle=\color{redstrings},
  showtabs=false,
  breaklines=true,
  breakatwhitespace=true,
  numberstyle=\footnotesize\ttfamily,
  numbers=left,
  stepnumber=3,
  breaklines=true
  escapeinside={(*@}{@*)},
  tabsize=2                       % sets default tabsize to 2 spaces
%  title=\lstname                   % show the filename of files included with \lstinputlisting; also try caption instead of title
}


\title{01435 Practical Cryptanalysis\\Project 2}
\author{Kim Rostgaard Christensen - s084283}
%\author{Thomas Pedersen - s103665}
%\author{Lars Kristensen - s072662}
\begin{document}
\maketitle
\begin{abstract}
The purpose of this report is to document the tool developed for generation and testing of Rainbow tables using a practical simulated attack.\\
Full sources can be found at:\\
 \url{https://github.com/rostgaard/practical_cryptoanalysis/}
\end{abstract}

\section*{Project description}
The program developed here is a small simulation of a man-in-the-middle attack with the purpose of making a communicating party reveal it's private key. For this, rainbow tables are used. The simulation is realized by simulating a fob for unlocking a car - and responding on behalf of the car.
\section*{Design and implementation}
To get the secret out of the fob, we generate a Non-Perfect Rainbow table based on the fixed $U$ we send to it.
The code implementing a serializable Rainbow table - with generation - is shown in figure \ref{fig:rainbow}. The serialization enables us to save and load generated tables - thus saving computation time across software runs.\\
We use a hashed map that contains sets of values, so we can store multiple start values for the same endpoint.
The idea when we generate the rainbow table is that we have a random secret, as start value, but the value we parse to MD5, is the secret concatenated with a fixed challenge $U$. That way, we have the same hash value as we would get from the fob, but the key that is saved, is only the secret. Of course when we generate a chain in the rainbow table, we also use a different reduction-function on each hash value. So; $S \rightarrow MD5(S ||U) \rightarrow  Hash \rightarrow  ReductionFunction  - N$.
\begin{figure}[h]
\begin{lstlisting}
public class RainbowTable extends    HashMap<Long, HashSet<Long>> 
                          implements Serializable {
                          
  public void generate(long u, long rows, long chainlength) {
    this.u           = u;
    this.rows        = rows
    this.chainlength = chainlength;
    
    SecureRandom ran = new SecureRandom();
    long     bitMask = 0xfffff;

    for (int i = 0; i < this.rows; i++) {
        long startValue = ran.nextInt() % bitMask;
        long accumulator = startValue;
        
        for (int j = 0; j < this.chainLength; j++) {
          long cipher = Utilities.MD5_Hash(Utilities.combine(accumulator, this.u, bitMask), bitMask);
          long reducedCipher = Utilities.reductionFunction(cipher, j, bitMask + 1);
          accumulator = reducedCipher;
        }

        this,put(accumulator, startValue);
      }
  }
}
\end{lstlisting}
\caption{Code for generating a rainbow table.}
\label{fig:rainbow}
\end{figure}
\section*{Usage and test}
When the program starts, it generates a Rainbow table - if not already present. It then uses the Rainbow table to start a simulation
We get a success rate of around 50\%
\begin{small}

\begin{verbatim}
\end{verbatim}
\end{small}

\section*{Further work}
\end{document}