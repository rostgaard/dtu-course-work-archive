\chapter[Requirements]{Requirements analysis and specification}
\label{chap:requirements}
%ELSAM states that we should be sensitive to changes in the powergrid frequency, and be able to communicate using a network.
The ELSAM agreement states that the grid frequency at all times must be 50Hz. If the values goes lower, our device should turn off unneeded devices. ELSAM defines two types of loads; Normal operation reserve and Disturbance reserve.\\
The Normal operation reserve should off-load when the frequency drops 49.9Hz, and shut off equipment able to do so in 2-3 minutes or less, not crucial to production (e.g. heater or lighting).\\ 
Disturbance reserve mode is used when the grid frequency does not return to normal state. In our case we turn off a battery powered device (a laptop) and try to make sure we dont empty their batteries completely before getting the power back.\\\\

To be able to do more ``intelligent'' grid-off loading, we must be able to communicate with the outside world. As there is no protocol specified, the logical choice would be tcp/ip for transport, as there is is already a well-established global infrastructure using this. The data should be transferred in an easy parsable format, such as XML.\\\\
Users of the device should also be able to view the status, current configuration and make changes to the configuration as well. The status and configuration should be available both from a physical interface and via a remote interface. As out device has a touchscreen it will serve the purpose as a physical interface, and the remote interface should be done in html.\\\\
This leaves us with the following requirements:
\begin{itemize}
\item Detect grid frequency, and respond to changes
\item Toggle relays based on frequency algorithm
\item Implement a TCP/IP stack
\item Implement a HTTP server
\item Design human interfaces for local access
\item Design human interfaces for remote access
\item Design machine interfaces for remote access
\end{itemize}

\section{Tools used}
\subsection{Debugging tools}
When developing the communications interface we used the program ``wireshark'' \footnote{www.wireshark.org} to verify http requests and responses.
% Optional paragraph containing descriptions on svn, wiki, debugging methods aso. 
\subsection{Versioning}
When developing an application, a versioning system is very useful both for experimenting with new features, because you have the ability to quickly roll-back to a working version. It is also a great way to do backup on your project.

%\section{Chapter Summary}  %% Should this be here ?
%\label{sec:SummaryChap2}



