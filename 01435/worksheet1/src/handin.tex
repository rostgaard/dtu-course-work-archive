\documentclass[10pt,a4paper]{article}
\usepackage[utf8]{inputenc}
\usepackage{amsmath}
\usepackage{amsfonts}
\usepackage{amssymb}
\usepackage{listings}

\usepackage{color}

\def\File#1{\textsf{#1}}
\def\Code#1{\texttt{#1}}
\def\Key#1{\textsf{#1}}

\definecolor{mygreen}{rgb}{0,0.6,0}
\definecolor{mygray}{rgb}{0.5,0.5,0.5}
\definecolor{mymauve}{rgb}{0.58,0,0.82}

\lstset{ %
  backgroundcolor=\color{white},   % choose the background color; you must add \usepackage{color} or \usepackage{xcolor}
  basicstyle=\footnotesize,        % the size of the fonts that are used for the code
  breakatwhitespace=false,         % sets if automatic breaks should only happen at whitespace
  breaklines=true,                 % sets automatic line breaking
  captionpos=b,                    % sets the caption-position to bottom
  commentstyle=\color{mygreen},    % comment style
  deletekeywords={...},            % if you want to delete keywords from the given language
  escapeinside={\%*}{*)},          % if you want to add LaTeX within your code
  extendedchars=true,              % lets you use non-ASCII characters; for 8-bits encodings only, does not work with UTF-8
  frame=single,                    % adds a frame around the code
  keywordstyle=\color{blue},       % keyword style
  language=Octave,                 % the language of the code
  morekeywords={*,...},            % if you want to add more keywords to the set
  numbers=left,                    % where to put the line-numbers; possible values are (none, left, right)
  numbersep=5pt,                   % how far the line-numbers are from the code
  numberstyle=\tiny\color{mygray}, % the style that is used for the line-numbers
  rulecolor=\color{mygray},         % if not set, the frame-color may be changed on line-breaks within not-black text (e.g. comments (green here))
  showspaces=false,                % show spaces everywhere adding particular underscores; it overrides 'showstringspaces'
  showstringspaces=false,          % underline spaces within strings only
  showtabs=false,                  % show tabs within strings adding particular underscores
  stepnumber=2,                    % the step between two line-numbers. If it's 1, each line will be numbered
  stringstyle=\color{mymauve},     % string literal style
  tabsize=2                       % sets default tabsize to 2 spaces
%  title=\lstname                   % show the filename of files included with \lstinputlisting; also try caption instead of title
}

\title{01435 Practical Cryptanalysis}
%\author{Kim Rostgaard Christensen - s084283}
\author{Thomas Pedersen - s103665}
\begin{document}
\maketitle

%\paragraph{Exercise 1:} Decrypt the following ciphertext that was encrypted with a Caesar cipher
%\1begin{center}
%GJBFW JYMJN IJXTK RFWHM
%\end{center}
%Output from program \Code{exercise\_01.adb}:

%\lstinputlisting[frame=single] {../output/exercise_01.output}

%\paragraph{Exercise 2:}
%The ciphertext ``ABQZ'' can translate to either ``STIR'' or ``OPEN''.

%9.000+ was found by means of brute force and an English dictionary.

\paragraph{Exercise 3:}
Known plaintext is a special case of chosen plaintext.
\begin{description}
\item[Known plaintext attacks:]
The simple caesar cipher, can be broken only knowing a sigle letter, but a more real life example is WEP(Wired Equivalent Privacy).\\
It has been shown that it can be broken, with a passive attack but that requires thousands of packages.\\
In 2006 Bittau, Handly and Lackey showed that by only capturing a single package, followed by sending 128 packages, it is possible to get the key.\footnote{Source: http://en.wikipedia.org/wiki/Wired\_Equivalent\_Privacy}

\item[chosen plaintext attacks:] If we look at the Vigenère cipher, that one can easily be broken if you choose a text that is at least the key lengh, which only consists of the same letter, for instance: if the Key is "HALLO" then we could choose the plaintext "AAAAA", which would be encrypted to: HALLO. and just like that we get the key.\footnote{Source: http://en.wikipedia.org/wiki/Vigen\%C3\%A8re\_cipher
}

\end{description}


%Source: 



%\paragraph{Exercise 4:}
\paragraph{Exercise 5:}
%In one hour 11.385.500.000 keys tested.

For this implementation, 2.160.000.000 keys was tested in an hour.\\\\
This gives $600000 \cdot 24 \cdot 365 = 18.921.600.000.000 \approx 19 \cdot 10^{12}$ in a year.\\\\
Given a cluster of 1 million machines, the number would be $19 \cdot 10^{12} \cdot 10^6 = 19 \cdot 10^{18}$ keys attempted, roughly corresponding to ~64 bit.\\\\
Code is listed in figure \ref{lst:exercise_05}.

\begin{figure}
\lstinputlisting[frame=single,language=Ada] {../src/exercise_05.adb}
\caption{exercise\_05.adb}
\label{lst:exercise_05}
\end{figure}

\section*{Exercise 6}
For $|K|$, our keyspace is $26!$, $R_L$ is $26$, which gives us;
\begin{equation}
UD \approx \frac{\log_2{26!}}{0.7 \cdot \log_2{26}} \Leftrightarrow UD \approx 26.7
\end{equation}

%\section*{Exercise 7}
%\section*{Exercise 8}
\section*{Exercise 9}
Our test-population is 1000, given that we have 1000 \emph{pairs}.\\
Feeding the this to the $\chi^2$ test gives the following;
\begin{eqnarray}
\chi^2 = \frac{\left(231 - 500\right)^2}{250} + 
         \frac{\left(271 - 500\right)^2}{250} + 
         \frac{\left(270 - 500\right)^2}{250} + 
         \frac{\left(228 - 500\right)^2}{250} \\
\chi^2 = 6.744
\end{eqnarray}

This is then fed to R, which gives out:

\begin{eqnarray}
CHIDIST \left(6.744,3\right) = 0.08
\end{eqnarray}

Around 8\% of the experiments will result in a distribution similar to this. We should not reject this as acceptable output, as it is within the standard 95\% confidence interval. Since the probability is so low, more data should be acquired to get more accurate results.

%\section*{Exercise 10}
\section*{Exercise 11}
Putting the table into a functional graph, we observe that it consists of two disjoint cycles. One of them being merely $4 \rightarrow 1 \rightarrow 4$.
When using the input vector consisting of identical elements (all 3's), we observe the function repeating itself after 14 iterations.\\
The only exception is when using 9 as initialization vector. This gives us a loop of one, repeating 9 as input for the next element.

\end{document}