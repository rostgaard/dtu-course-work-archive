\documentclass[12pt]{article}

\usepackage[T1]{fontenc}
\usepackage[scaled]{beramono}
%\usepackage{bera}

\usepackage{libertine}
\usepackage[T1]{fontenc}

\usepackage{color}
\definecolor{bluekeywords}{rgb}{0.13,0.13,1}
\definecolor{greencomments}{rgb}{0,0.5,0}
\definecolor{redstrings}{rgb}{0.9,0,0}

\definecolor{mygreen}{rgb}{0,0.6,0}
\definecolor{mygray}{rgb}{0.2,0.2,0.2}
\definecolor{mymauve}{rgb}{0.58,0,0.82}

\usepackage{listings}
%\lstset{
%}

\lstset{ %
  language=[Sharp]C,
  backgroundcolor=\color{white},   % choose the background color; you must add \usepackage{color} or \usepackage{xcolor}
  showspaces=false,
  basicstyle=\footnotesize\ttfamily,  % the size of the fonts that are used for the code
  breakatwhitespace=false,         % sets if automatic breaks should only happen at whitespace
  breaklines=true,                 % sets automatic line breaking
  captionpos=b,                    % sets the caption-position to bottom
  deletekeywords={...},            % if you want to delete keywords from the given language
  escapeinside={\%*}{*)},          % if you want to add LaTeX within your code
  extendedchars=true,              % lets you use non-ASCII characters; for 8-bits encodings only, does not work with UTF-8
  frame=single,                    % adds a frame around the code
%  keywordstyle=\color{blue},       % keyword style
%  language=Octave,                 % the language of the code
  morekeywords={*,...},            % if you want to add more keywords to the set
  numbers=left,                    % where to put the line-numbers; possible values are (none, left, right)
  numbersep=5pt,                   % how far the line-numbers are from the code
  numberstyle=\tiny\color{mygray}, % the style that is used for the line-numbers
  rulecolor=\color{mygray},         % if not set, the frame-color may be changed on line-breaks within not-black text (e.g. comments (green here))
  showspaces=false,                % show spaces everywhere adding particular underscores; it overrides 'showstringspaces'
  showstringspaces=false,          % underline spaces within strings only
  showtabs=false,                  % show tabs within strings adding particular underscores
  stepnumber=2,                    % the step between two line-numbers. If it's 1, each line will be numbered
  commentstyle=\color{greencomments},
  keywordstyle=\color{bluekeywords}\bfseries,
  stringstyle=\color{redstrings},
  showtabs=false,
  breaklines=true,
  breakatwhitespace=true,
  numberstyle=\footnotesize\ttfamily,
  numbers=left,
  stepnumber=3,
  breaklines=true
  escapeinside={(*@}{@*)},
  tabsize=2                       % sets default tabsize to 2 spaces
%  title=\lstname                   % show the filename of files included with \lstinputlisting; also try caption instead of title
}

\title{01435 Practical Cryptanalysis}
\author{Kim Rostgaard Christensen - s084283}
\begin{document}
\maketitle
 

\section{Exercise 19 (P)}

With the set of English letters, $S$, and the corresponding set of letter-probabilities, $P$, we compute the Entropy, $H$, using the function\\

\begin{equation}
H(P) = \sum_{i=1}^{n} p_i \times log_2(\frac{1}{p_i})
\end{equation}
where $p_i$ is the probability of the corresponding letter $s_i$, to occur in a text written in the English language.\\
To compute the Entropy, a C\#-program was written. The letter-probabilities from the lecture slides are paired with the corresponding letter in a Dictionary-datatype. The computation is then done for each entry in the dictionary, and summed in an entropy-value.\\
Using this approach, we compute the Entropy to be approximately $4.18$. This number indicates the average number of bits needed to store one letter of an English plaintext. \\\\
The code used to calculate the entropy, is found on figure \ref{lst:code}.

\begin{figure}

\begin{lstlisting}
using System;
using System.Collections.Generic;

namespace Cryptanalysis.Entropy {
  class Program {
    static void Main(string[] args) {
      //Probabilities extracted from the lecture slides.
      Dictionary<char, double> probabilities = new Dictionary<char, double>() {
                {'A', 0.082F}, {'B', 0.015F},
                {'C', 0.028F}, {'D', 0.043F},
                {'E', 0.127F}, {'F', 0.022F},
                {'G', 0.020F}, {'H', 0.061F},
                {'I', 0.070F}, {'J', 0.002F},
                {'K', 0.008F}, {'L', 0.040F},
                {'M', 0.024F}, {'N', 0.067F},
                {'O', 0.075F}, {'P', 0.019F},
                {'Q', 0.001F}, {'R', 0.060F},
                {'S', 0.063F}, {'T', 0.091F},
                {'U', 0.028F}, {'V', 0.010F},
                {'W', 0.023F}, {'X', 0.001F},
                {'Y', 0.020F}, {'Z', 0.001F}
      };
      double Entropy = 0.0F;

      for (char c = 'A'; c <= 'Z'; c++) {
        Entropy += probabilities[c] * Math.Log((1 / probabilities[c]), 2);
      }

      Console.WriteLine("Entropy: " + Entropy);
      //Entropy: 4,18024503236223
    }
  }
}
\end{lstlisting}
\label{lst:code}
\caption{C\# program used for calculating entropy}
\end{figure}
\end{document}