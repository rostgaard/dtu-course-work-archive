\documentclass[presentation]{beamer}   % to compile the presentation
% \documentclass[handout]{beamer}        % to compile 2x2 handouts
\usepackage[ansinew]{inputenc}
\usepackage[T1]{fontenc}
\usepackage{lmodern,textcomp}
\usepackage{breakurl}

\usepackage{listings}
\usetheme{dtu}

\begin{document}

% The DTU and MIC logos
\pgfdeclareimage[height=2cm]{dtulogo}{dtu_logo}
\pgfdeclareimage[width=0.8\framesep]{miclogo}{mic_logo}

\author{Kim Rostgaard Christensen}
\title{SPARK - correctness by construction}

\date{Nov 30 2010}
\institute[%
  Dept. of Informatics and Mathematical Modelling
  ]{%
  DTU -- Technical University of Denmark \\
}
% \titlegraphic{\pgfuseimage{dtulogo}} % Graphics for title slide
%\logo{\pgfuseimage{miclogo}} % The left logo

\begin{frame}
  \maketitle
\end{frame}

\begin{frame}
  \frametitle{Outline}
  \tableofcontents
\end{frame}

\section{Correctness by construction}
\begin{frame}
  \frametitle{Correctness by construction}
  \framesubtitle{Putting engineering back into software engineering}
  \begin{itemize}
    \item CbyC challenges testing and debugging
    \item Proves the software is correct
    \item Introduces a formalism
      \begin{itemize}
        \item Elaborates requirements 
        \item Design-by-contract
      \end{itemize}
  \end{itemize}
\end{frame}

\section{Design-by-contract}
\begin{frame}
  \frametitle{Design-by-contract?}
  \framesubtitle{The fine print}
  \begin{itemize}
    \item Translate requirements to a formal language
    \item Use this as the specification
    \item Hold the implementation to this
	\item Use automated tools
      \begin{itemize}
        \item Not subject to human error
      \end{itemize}
    \item Upon validation failure, either
      \begin{itemize}
        \item The implementation is wrong (potentially)
        \item The specification is wrong, redesign
      \end{itemize}  
  \end{itemize}
\end{frame}

\section{Example}
\begin{frame}
  \frametitle{Example contract}
  \framesubtitle{What does it look like}

\end{frame}

\section{Proving correctness}
\begin{frame}
  \frametitle{Proving correctness}
  \framesubtitle{Quod erat demonstrandum}
   \begin{itemize}
        \item Static analysis
        \begin{itemize}
         \item Data flow analysis
         end{itemize}
        \item Constructs hypothesis and conclusions, bottom up
      \end{itemize}  
\end{frame}

\section{Gains}
\begin{frame}
  \frametitle{Gains}
  \framesubtitle{The good, the bad and the ugly}
  \begin{itemize}
        \item Formal and systematic approach to development
        \item Better control
        \item Proof == 100\% correctness (according to specification)
        \item Forces you to spend time on design
        \item Dictates modularity
        \item Improves maintainability
      \end{itemize}       
\end{frame}

\section{Challenges}
\begin{frame}
  \frametitle{Challenges}
  \framesubtitle{The good, the bad and the ugly}
  \begin{itemize}
        \item Formal and systematic approach to development
        \item Better control of resources (time, money)
        \item Proof == 100\% correctness (according to specification)
        \item Forces you to spend time on design
        \item Dictates modularity
        \item Improves maintainability
      \end{itemize}       
\end{frame}


\section{Design-by-contract}
\begin{frame}
  \frametitle{Design-by-contract?}
  \begin{columns}[t] % Align the columns at the top
    \column{0.4\textwidth}
      This is the \alert{first} column. It occupies $40$\% of the text width.
    \column{0.6\textwidth}
      This is the \alert{second} column. This could be a nice image\ldots
      \begin{center}
        \rule{0.4\textwidth}{0.3\textwidth}
      \end{center}
  \end{columns}
\end{frame}

\section{Blocks}
\begin{frame}
  \frametitle{Use blocks to highlight your points}
  \begin{block}{<The title of the block>}
    This is the point you want to highlight. It could be an important formula
    \[
      a^2+b^2=c^2
    \]
  \end{block}

  \begin{example}
    The example block is useful for typesetting examples consistently.
  \end{example}
\end{frame}

\section{Verbatim}
\begin{frame}[fragile]
  \frametitle{Verbatim material}
  If the slide contains verbatim material you must use the \texttt{fragile} option for the frame.

  \begin{verbatim}
  This is verbatim text
  !"#%&/()=?
  \end{verbatim}
  
  The \texttt{listings} package can be used for more fancy verbatim text and pretty printing of source code.
\end{frame}

\section{Finding documentation}
\begin{frame}
  \frametitle{Beamer documentation}
  The Beamer userguide is available on-line at CTAN:
  \begin{center}
    \url{ftp://tug.ctan.org/pub/tex-archive/macros/latex/contrib/beamer/doc/beameruserguide.pdf}
  \end{center}

  If Beamer is installed on your system you can find the manual by running
  \begin{center}
    \texttt{mthelp beamer}
  \end{center}
  in a Command Prompt.
\end{frame}

\section{DTU stuff}

\begin{frame}
  \frametitle{The template}
  This presentation template is a \texttt{beamer} implementation of the official DTU PowerPoint template available at
  \begin{center}
    \url{http://portalen.dtu.dk/Services/Kommunikation.aspx}
  \end{center}
  
  To follow the design guidelines completely, you should use the colors from the DTU color palette
  \begin{center}
    \url{http://portalen.dtu.dk/upload/ak/design/dtu-farvemanual_03_07_2006.pdf}
  \end{center}
  These colors are defined in the DTU beamer theme with the names shown on the next slide.
\end{frame}
  
  \newcommand\dtucolorbox[1]{\parbox[b][1cm+2ex][c]{2cm}{\tiny \centering\color{#1}\rule{.8cm}{.8cm}\\ \textcolor{black}{#1}}}%
\begin{frame}
  \frametitle{DTU colors}
  \centering
  \dtucolorbox{dtudarkgray}\dtucolorbox{dtugray}\dtucolorbox{dtulightgray}\\
  \dtucolorbox{dtudarkblue}\dtucolorbox{dtublue}\dtucolorbox{dtulightblue}\\
  \dtucolorbox{dtured}\dtucolorbox{dtupurpur}\dtucolorbox{dtupurple}\\
  \dtucolorbox{dtudarkorange}\dtucolorbox{dtuorange}\dtucolorbox{dtuyellow}\\
  \dtucolorbox{dtudarkgreen}\dtucolorbox{dtugreen}\rule{2cm}{0pt}\\
\end{frame}

\end{document} 