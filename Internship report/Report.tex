%Rapporten har to formål:
%
%
%1.	At give en orientering til DTU-vejlederen om det faglige indhold i praktikken. Dette er gældende også selvom du ikke laver eksamensprojekt i praktikvirksomheden.  Tal med din DTU-vejlederen og lad praktikrapporten være med til at afstemme de fælles forventninger til eksamensprojektet. 
%
%2.	At give en tilbagemelding til praktikkoordinatorerne på den studerendes vurdering af virksomheden som praktikplads. Rapporten læses af praktikkoordinatorerne og arkiveres herefter til senere brug. Tilbagemeldingen anvendes såvel på kort som lang sigt til vurdering af den enkelte virksomheders egnethed som praktikværter.
%Ønsker den studerende at bidrage med mundtlig tilbagemelding, f.eks. om ting som er mere velegnet for en 2 vejs mundtlig kommunikation, skal praktikkoordinatorerne kontaktes. 
%
%Rapporten forventes at have et omfang på mellem 5 og 10 sider.
%
%Du skal være opmærksom på, at arbejde udført i praktikperioden ikke direkte kan indgå som en del af eksamensprojektet. Eksamensprojektet omfatter udelukkende de aktivite-ter der udføres i denne periode.
%Ønsker du at referere til resultater fra praktikperioden, skal det gøres ved at etablere re-sultater fra praktikperioden som rapporter, artikler eller andre former for afsluttede do-kumenter der har dig som forfatter. Disse dokumenter kan du så referere til som bilag til eksamensprojektet.


\documentclass[11pt,a4paper,UKenglish]{article}
% Template by DTU LaTeX support group, v20090423

\usepackage[latin1]{inputenc} % Must correspond to the input encoding used by the editor
\usepackage{babel} % Other languages, in this case UKenglish
\usepackage[T1]{fontenc} % font encoding (output), use T1 for most latin languages
\usepackage{lmodern} % vector based Computer Modern font
\usepackage{graphicx} % for graphics
\graphicspath{{./figures/}} % path to figures and images

\usepackage{mathtools} % mathmatics

\usepackage{xcolor}
\usepackage[plainpages=false,pdfpagelabels,pageanchor=false]{hyperref} % active links
\hypersetup{
  pdfauthor={Kim Rostgaard Christensen},
  pdftitle={Internship report},
  pdfsubject={Internship report},
  citebordercolor={1 1 1}, % The color of the box around citations (change to 1 1 1 to remove)
  linkbordercolor={1 1 1}, % The color of the box around normal links
  pagebordercolor={1 1 1}, % The color of the box around links to pages
  urlbordercolor= {1 1 1}  % The color of the box around links to URLs
}
\usepackage{memhfixc}% fixes for hyperref

% The following can be used if not using custom frontpage
\title{Internship report}
\author{Kim Rostgaard Christensen\\
        \\
        s084283
}
\date{\today}


% \includeonly{preface,assignments,testing} % Only compile those files you are working in (to save compile time, and power)

\begin{document}
%\frontmatter % roman page numbering
\maketitle  % use this if not using custom frontpage
%\begin{titlingpage}
\centering \parindent=0pt
\newcommand{\HRule}{\rule{\textwidth}{1mm}}
\vspace*{\stretch{1}} \HRule\\[1cm]\Huge\bfseries
Internship report\\[0.7cm]
\large Atkins Denmark\\[1cm]
\HRule\\[4cm]  
\large by Kim Rostgaard Christensen, s084283\\
\vspace*{\stretch{2}} \normalsize %
\begin{flushleft}
Technical University of Denmark\\
Institute for Mathematical Modelling\\
Internship at Atkins Denmark\\
Per Stoltze\\
\today \end{flushleft}
\end{titlingpage}


\begin{abstract}
This report document my three-months internship with Atkins Denmark. The overall goal of the internship period was to learn about the railway industry with a focus on safety engineering.
\end{abstract}

\tableofcontents
\listoffigures
\listoftables

%\mainmatter % arabic page numbering
\section*{Introduction} 
As part of my engineers degree, it is mandatory for me to have an internship period of 20 weeks.

My internship period is reduced from 20 weeks to 3 (rounded from 10 weeks months due to received merits.\\\\
The defined purpose of the stay was for me to study the theory behind 



\section{Atkins Denmark}
This section contains a brief description of Atkins Denmark, its role in the Atkins Global corporation and the Risk Management department where I was located.


Atkins Denmark is an advisory company that operates within the following areas of expertise:

\begin{itemize}
  \item Transportation - mainly railway
  \item Climate and environment
  \item GIS \& IT
  \item Surveying
  \item Risk Management
  \item Energy
  \item Architecture \& Design
  \item Bridges \& Construction
\end{itemize}
\subsection{Risk Department}
The risk department where I worked is again split into two branches; RAMS and Validation.\\\\
The RAMS department is not surprisingly doing RAMS\footnote{Reliability Availability Maintainability Safety} tasks. Furthermore they do risk assessment, qualitative and quantitative analyses and modelling, including methods design. Last, but not least the also do Human Factors analysis\\\\
The validation department is in charge of validating technical systems, primarily interlocking systems. The department has approved in-house validators.\\\\
Both branches work primarily with railway safety, which is also considered their core competence.


\section{Assignments}

\subsection{Railway theory and business management}
Although technically not an assignment, I have decided to include this as part of my internship report. This i mainly due to the fact, that I have spent i great deal of time studying the concepts and application of railway technology and application.

I have taken the following two courses during my stay at Atkins.

\begin{itemize}
  \item Introductory railway concepts
  \item Introduction to Business management system
\end{itemize}


\subsection{Document and technical drawing archiving}
but was planned to be a holistic digital archiving solution for paper documents and drawings.

The concept was to have a three-phase storage process
\begin{itemize}
  \item Scan the document
  \item Archive the document on a net share
  \item Use OCR to auto-tag documents - or provide fulltext search indexes.
\end{itemize}
This project never left the theoretical discussion stage, unfortunately.

\subsection{Defining and documenting a V\&V process}
This project was the main time-consumer of my time. It is a large project, already in progress, which meant that there was a vast quantity of documentation to read through.

This task involved reading though the existing documentation - which was extensive. And Modelling. See section \ref{sec:vandv} for more details


\subsection{Learn concepts and application of FRAM}
The Functional Resonance Analysis Method - or FRAM - is a method developed by Erik Hollnagel. It is a radically different approach to resilience engineering, taking into account human factors as functions in a system.


\section{V\&V application}
\subsection{Screenshots}
Always have images/figures in \emph{both} EPS and PDF/PNG/JPG format! (unless you know you only will be using pdfLaTeX)


\begin{figure}[!tbh]
	\centering
	\includegraphics[width=1.0\textwidth]{img/login.png}
	\caption{Main login screen}
	\label{fig:login}
\end{figure}

\begin{figure}[!tbh]
	\centering
	\includegraphics[width=1.0\textwidth]{img/frontpage.png}
	\caption{Main login screen}
	\label{fig:login}
\end{figure}


\section{Conclusion}


\subsection{Criticism}
13 weeks has bee too short of a period for an internship. It took me about 5 weeks to get comfortable enough with the internal work flows, to be able to have a complete working day.

\include{biblio/bib}

\appendix % Alphabetic chapter numbering
%\renewcommand{\appendixtocname}{Appendix} % Change appendix name for the table od contents
%\addappheadtotoc % 'Appendix' to the table of contents



%\backmatter % for glossary, index, back page
\end{document}
