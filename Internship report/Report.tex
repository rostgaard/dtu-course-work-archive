%Rapporten har to formål:
%
%
%1.	At give en orientering til DTU-vejlederen om det faglige indhold i praktikken. Dette er gældende også selvom du ikke laver eksamensprojekt i praktikvirksomheden.  Tal med din DTU-vejlederen og lad praktikrapporten være med til at afstemme de fælles forventninger til eksamensprojektet. 
%
%2.	At give en tilbagemelding til praktikkoordinatorerne på den studerendes vurdering af virksomheden som praktikplads. Rapporten læses af praktikkoordinatorerne og arkiveres herefter til senere brug. Tilbagemeldingen anvendes såvel på kort som lang sigt til vurdering af den enkelte virksomheders egnethed som praktikværter.
%Ønsker den studerende at bidrage med mundtlig tilbagemelding, f.eks. om ting som er mere velegnet for en 2 vejs mundtlig kommunikation, skal praktikkoordinatorerne kontaktes. 
%
%Rapporten forventes at have et omfang på mellem 5 og 10 sider.
%
%Du skal være opmærksom på, at arbejde udført i praktikperioden ikke direkte kan indgå som en del af eksamensprojektet. Eksamensprojektet omfatter udelukkende de aktivite-ter der udføres i denne periode.
%Ønsker du at referere til resultater fra praktikperioden, skal det gøres ved at etablere re-sultater fra praktikperioden som rapporter, artikler eller andre former for afsluttede do-kumenter der har dig som forfatter. Disse dokumenter kan du så referere til som bilag til eksamensprojektet.




\documentclass[11pt,a4paper,UKenglish]{memoir}
% Template by DTU LaTeX support group, v20090423

\usepackage[latin1]{inputenc} % Must correspond to the input encoding used by the editor
\usepackage{babel} % Other languages, in this case UKenglish
\usepackage[T1]{fontenc} % font encoding (output), use T1 for most latin languages
\usepackage{lmodern} % vector based Computer Modern font
\usepackage{graphicx} % for graphics
\graphicspath{{./figures/}} % path to figures and images

\usepackage{mathtools} % mathmatics

\usepackage{xcolor}
\usepackage[plainpages=false,pdfpagelabels,pageanchor=false]{hyperref} % active links
\hypersetup{%
  pdfauthor={<your name>},
  pdftitle={title},
  pdfsubject={<subject>},
  citebordercolor={1 1 1}, % The color of the box around citations (change to 1 1 1 to remove)
  linkbordercolor={1 1 1}, % The color of the box around normal links
  pagebordercolor={1 1 1}, % The color of the box around links to pages
  urlbordercolor= {1 1 1}  % The color of the box around links to URLs
}
\usepackage{memhfixc}% fixes for hyperref

% The following can be used if not using custom frontpage
\title{<document title>}
\author{<name>\\
        \\
        <student id>
}
\date{\today}

% \includeonly{preface,testing} % Only compile those files you are working in (to save compile time, and power)

\begin{document}
\frontmatter % roman page numbering
%\maketitle  % use this if not using custom frontpage
\begin{titlingpage}
\centering \parindent=0pt
\newcommand{\HRule}{\rule{\textwidth}{1mm}}
\vspace*{\stretch{1}} \HRule\\[1cm]\Huge\bfseries
Internship report\\[0.7cm]
\large Atkins Denmark\\[1cm]
\HRule\\[4cm]  
\large by Kim Rostgaard Christensen, s084283\\
\vspace*{\stretch{2}} \normalsize %
\begin{flushleft}
Technical University of Denmark\\
Institute for Mathematical Modelling\\
Internship at Atkins Denmark\\
Per Stoltze\\
\today \end{flushleft}
\end{titlingpage}


\begin{abstract}
This report document my three-months internship with Atkins Denmark. The internship is part of my education, and is reduced from 20 weeks to 3 months due to received merits.
\end{abstract}

\tableofcontents
\listoffigures
\listoftables

\mainmatter % arabic page numbering
\chapter*{Introduction} 
As part of my engineers degree, it is mandatory for me to \dots


\chapter{Assignments}

\section{Railway theory}
Although technically not an assignment, I have decided to include this as part of my internship report. This i mainly due to the fact, that I have spent i great deal of time studing the concepts and application of railway technology and application.

I have taken a course in "introductory railway concepts" during my stay at Atkins.


\section{Document and technical drawing archiving}

This project never left the theoretical discussion stage.

\section{Defining and documenting a V\&V process}
This project was the main time-consumer of my time. It is a large project, already in progress, which meant that there was a vast quantity of documentation to read through.

This task involved reading though the existing documentation - which was extensive. And Modelling. See \cite{Rostgaard_VandV:2011}



\section{Learn concepts and application of FRAM}

\include{testing}
\chapter{Conclusion}
\label{chap:conclusion}
In the beginning of the project we spent a great deal of time cleaning and refactoring code, making it more modulary and easy to navigate. But as the application evolved and grew in size and complexity, it became increasingly difficult to maintain a consistency in the code. Espeacially when thing were not doing as planned. New code arose in places where it did not belong.\\
From this we have learned about the importance of defining a clear program structure and functionality from the start of the project, and enforcing it thoughout the development process.\\\\
The webserver is responding slowly, but is very functional. The xsl transformations works like a charm in modern browsers and should definitly be considered in the case of an official http+xml implementation.\\\\
The LCD screen, acting as both in- an output, gives a quick way of getting an overview of the current status of the system. As well as modifying the system paramters realtime.\\ The values displayed are a bit off, due to programming errors. But it gives you an idea on how a finished system should look and behave.


\include{biblio/bib}

\appendix % Alphabetic chapter numbering
\renewcommand{\appendixtocname}{Appendix} % Change appendix name for the table od contents
\addappheadtotoc % 'Appendix' to the table of contents
\include{appendix-code}
\backmatter % for glossary, index, back page
\end{document}
