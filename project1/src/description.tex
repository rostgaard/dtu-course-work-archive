\documentclass[10pt,a4paper]{article}
\usepackage[utf8]{inputenc}
\usepackage{url}
\usepackage{amsmath}
\usepackage{amsfonts}
\usepackage{amssymb}
\usepackage{listings}
\usepackage{graphicx}

\usepackage{color}

\def\File#1{\textsf{#1}}
\def\Code#1{\texttt{#1}}
\def\Key#1{\textsf{#1}}

\definecolor{mygreen}{rgb}{0,0.6,0}
\definecolor{mygray}{rgb}{0.5,0.5,0.5}
\definecolor{mymauve}{rgb}{0.58,0,0.82}

\lstset{ %
  backgroundcolor=\color{white},   % choose the background color; you must add \usepackage{color} or \usepackage{xcolor}
  basicstyle=\footnotesize,        % the size of the fonts that are used for the code
  breakatwhitespace=false,         % sets if automatic breaks should only happen at whitespace
  breaklines=true,                 % sets automatic line breaking
  captionpos=b,                    % sets the caption-position to bottom
  commentstyle=\color{mygreen},    % comment style
  deletekeywords={...},            % if you want to delete keywords from the given language
  escapeinside={\%*}{*)},          % if you want to add LaTeX within your code
  extendedchars=true,              % lets you use non-ASCII characters; for 8-bits encodings only, does not work with UTF-8
  frame=single,                    % adds a frame around the code
  keywordstyle=\color{blue},       % keyword style
  language=Octave,                 % the language of the code
  morekeywords={*,...},            % if you want to add more keywords to the set
  numbers=left,                    % where to put the line-numbers; possible values are (none, left, right)
  numbersep=5pt,                   % how far the line-numbers are from the code
  numberstyle=\tiny\color{mygray}, % the style that is used for the line-numbers
  rulecolor=\color{mygray},         % if not set, the frame-color may be changed on line-breaks within not-black text (e.g. comments (green here))
  showspaces=false,                % show spaces everywhere adding particular underscores; it overrides 'showstringspaces'
  showstringspaces=false,          % underline spaces within strings only
  showtabs=false,                  % show tabs within strings adding particular underscores
  stepnumber=2,                    % the step between two line-numbers. If it's 1, each line will be numbered
  stringstyle=\color{mymauve},     % string literal style
  tabsize=2                       % sets default tabsize to 2 spaces
%  title=\lstname                   % show the filename of files included with \lstinputlisting; also try caption instead of title
}

\title{01435 Practical Cryptanalysis\\Project 1}
\author{Kim Rostgaard Christensen - s084283}
%\author{Thomas Pedersen - s103665}
%\author{Lars Kristensen - s072662}
\begin{document}
\maketitle
\begin{abstract}
The purpose of this report is to document the tool developed for assisting decryption of simple substitution cipher encryption.\\
Full sources can be found at:\\
 \url{https://github.com/rostgaard/practical_cryptoanalysis/}
\end{abstract}

\section*{Project description}
The tool needs to read in two files;
a plain-text file used for generating a dictionary consisting of letters, digrams and trigrams, and a cipher-text file containing the data targeted for decryption.\\
Furthermore, it needs to, via user interaction, control substitutions and replacements of the substitution set.
\section*{Design and implementation}
For this project, a generic n-gram counter has been implemented. I counts the occurrences of unique n-grams and gives back a relative frequency on-demand. This enables us to easily instantiate a letter-counter (1-gram), bigram and digram-counters.\\
The program reads the two files, one character at the time, maintaining a circular buffer of the three latest characters for pushing into trigrams, bigrams and letters.
The menu is rather simple and responds to single key presses rather than key+enter. \\
Validation of correctness of decoding is done manually, and also noting down the table.

\section*{Usage}
Running the tool without parameters results in a usage description;
\begin{small}

\begin{verbatim}
Usage: substitution_cipher_decrypter plaintext_file (dictionary) 
                                     ciphertext_file [--debug]
\end{verbatim}
\end{small}
Feeding in the commands will result in a menu being presented after the n-grams has been created from file.

\section*{Further work}
Previously, in worksheet assignment 2, an aspell backend has been implemented to verify (dictionary) correctness of words. This could also be implemented in this tool to enable extend an a $\chi^2$ test.
The $\chi^2$ test is written, but not implemented as part of the tool. The probability distribution is not implemented either, but could be done by binding to \File{Rmath.h}.
\end{document}