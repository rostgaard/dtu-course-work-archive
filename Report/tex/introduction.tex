\chapter{Introduction}
\mrb

\noindent
In this report, we document and analyze web service technologies, and show our implementation of a travel agency. Different web service technologies are used for different parts of the system, with the main component being implemented twice, using two different technologies; BEPL and rest.

In this chapter, we go over some of the general web service concepts, followed by a chapter which documents how a client would interact with the web service(s). Afterwards we discuss the actual implementation of the system, as well as the decisions made in the process. We also take a look at how the web services used can be discovered using Web services inspection language. Finally we do a comparison of implementing web services using RESTful versus BPEL, as well as taking a look at some advanced web service technologies and how they could be applied to the project.

%This report documents the design and implementation of a travel agency service. The travel agency acts as a coordinator by composing and synchronizing bookings and accounts from third party actors, as well as handling payment transactions.

\section{Introduction to Web Services}
\mrb

\noindent
Web service technology uses the world wide web to transmit operation calls between computers. There are different protocols for it, but the actual implementation is not platform-specific. In this section, we take a look at some of the concepts that relates to web services and their role. 

 %-- which is a general term applied to services that use the world wide web and it's technologies as transportation layer -- and the general service orientation design pattern. Service orientation seeks to abstract away both implementation details and presentation from business logic in applications. The motivation for this is to provide easy modularization for use in composition in larger web services.

\subsection{Web Service Description}
\mkt

\noindent
Generally some sort of contract is created which describes the web service to clients that wish to use it. This contract can serve as an interface, allowing parties to agree on the contract such that the client never has to know how the results from the web service are actually achieved or even which language it is implemented with. One way to describe a web service, is the \emph{Web Service Definition Language} (WSDL). WSDL is based on \emph{XML} and XML schema to describe service ports and which operations there are on these ports. It is possible to write client code using a WSDL as a contract, much like an interface class in object oriented programming. This can be combined with Simple Object Access Protocol (SOAP) which describes how to encapsulate the data in an http request, in order to achieve the wanted web service.
%The following section is a description of some of the key technologies we will use in this project, along with a brief discussion on some of the objective strengths and weaknesses.

%Write a section in the introduction of the report introducing the basic concepts of Web services, such as
%service orientation, basic service technologies, description of Web services, Web service discovery, Web
%service composition, Web service coordination, and RESTful services. This section should paint the big
%picture and highlight the important aspects of each of these topics. What do you think is important and
%why? The size of the section should be around 2 pages.

\subsection{Service Orientation}
\mrb

\noindent
Service orientation is a paradigm for structuring systems. The system is partitioned into autonomous blocks that each provide a service. When the system runs, services depend on each other to perform tasks. When structuring a system like this using web services, each service can even be located at different places around the world, as the communications can go through the internet. When using a service-oriented architecture, the system becomes more modular and coupling is generally looser, as you are forced to condense concepts into highly coherent blocks, that can be combined to form a complete system. When we focus on system parts as services, we also make it easier to link the system to business processes, as these are often described in terms of services different departments might provide.

\subsection{Web service composition and coordination}
\kim

\noindent
Web service composition describes the dependency between different web services. Coordination describes when and how they actually interact. Languages, such as WDSL and BPEL exists to define the formal semantics of these relationships the web services have with each other as well as clients.

 \textbf{B}usiness \textbf{P}rocess \textbf{E}xecution \textbf{L}anguage is also an XML application designed for formalizing business processes as well as executing them. BPEL is flow-oriented and and concentrates on message flow, control flow, data flow and the fault and exception handling when things do not go right. It also orchestrates the different flows between services -- called partners -- by becoming a central point of authority. It is layered atop of WSDL and can be visualized graphically.

%\item[HTTP] The \textbf{H}yper\textbf{T}ext \textbf{T}ransfer \textbf{P}rotocol is a plaintext stateless request-response protocol with standardized %methods and response codes. It is deployed everywhere, from coffee machines to mainframes and thus a robust technology to base a web application on. Most programming languages have auxiliary libraries that provide both HTTP client and server components for re-use.

%\item[XML] The e\textbf{X}tensible \textbf{M}arkup \textbf{L}anguage is a subset of the Standard Generalized Markup Language (SGML) and has a very famous cousin named Hypertext Markup Language - or HTML. XML is, basically, a general purpose markup language for adding semantics to documents. This semantic is further formalized by either linking a Document Type Definition (DTD), or an XML Schema (XSL) to the document. A large advantage of XML is that it, like HTTP, is a widely used standard, and most programming languages have libraries for using XML without having to parse everything. XML - being a generalized language - has the further advantage that you can build new standards as XML applications and inherit the well-formedness\footnote{Grammatically and semantically correct.} property of of XML itself, and the general XML parser can used.

\subsection{Service discovery}
\pet

\noindent
Web services are deployed by various organization and businesses worldwide, and in order to find them a central directory -- similar to a phone book -- can provide easy querying of such services. Technologies such as \emph{UDDI} and \emph{WSIL}\footnote{Web Services Inspection Language} solve this for web services.

\textbf{U}niversal \textbf{D}escription, \textbf{D}iscovery and \textbf{I}ntegration is a registry service for centralizing information about web services and who provides provides them using XML in the form of WSIL documents. The UDDI is a also directory service providing information about the companies as well. The service providers are responsible for registering themselves into the registry, along with their WSDL specification.

\subsection{RESTful}
\mkt

\noindent
REST The concept \textbf{RE}presentational \textbf{S}tate \textbf{T}ransfer covers the architecture defined by Roy Fielding and is specific application to HTTP (1.1) and centralized around the concept of resources and representations\footnote{Typically states of resources}. Resources are accessed and changed by ordinary HTTP methods: GET, PUT, POST, DELETE and standard HTTP 1.1 response codes are used to explain to the caller how the request went. For instance, a DELETE operation on a non-existing ID could return a 404 - Not Found, while it would return 200 - OK if all went well. There exists no (standardized) formalization language to describe RESTful web services, like there does with WSDL.

%\item[JSON] \textbf{J}ava\textbf{S}cript \textbf{O}bject \textbf{N}otation is an object serialization language that is human-readable -- much like XML. One of the big advantages of JSON over XML is, however, that grammer, and thus effectively, the parser is very simple. It is also less verbose than XML and therefore has less overhead. It has draft support for schemas, and has currently no general way of validating data, or enforcing constraints. Due to its simple nature, it doesn't support any other types than the ones that are built-in. JSON is typically used in web applications, and in strong collaboration with the REST paradigm.