\chapter{Introduction}
%Write a section in the introduction of the report introducing the basic concepts of Web services, such as
%service orientation, basic service technologies, description of Web services, Web service discovery, Web
%service composition, Web service coordination, and RESTful services. This section should paint the big
%picture and highlight the important aspects of each of these topics. What do you think is important and
%why? The size of the section should be around 2 pages.

This report documents the design and implementation of a travel agency service The travel agency acts as a coordinator by composing and synchronizing bookings and accounts from third party actors, as well as handling payment transactions.

\section{Introduction to Web Services}


Web services orientation is a paradigm that couples web services -- which is a general term applied to services that use the world wide web and it's technologies as transportation layer -- and the general service orientation design pattern. Service orientation seeks to abstract away both implementation details and presentation from business logic in applications. The motivation for this is to provide easy modularization for use in composition in larger web services.\\\\
The following section is a description of some of the key technologies we will use in this project, along with a brief discussion on some of the objective strengths and weaknesses.

\begin{description}
\item[HTTP] The \textbf{H}yper\textbf{T}ext \textbf{T}ransfer \textbf{P}rotocol is a plaintext stateless request-response protocol with standardized methods and response codes. It is deployed everywhere, from coffee machines to mainframes and thus a robust technology to base a web application on. Most programming languages have auxiliary libraries that provide both HTTP client and server components for re-use.

\item[XML] The e\textbf{X}tensible \textbf{M}arkup \textbf{L}anguage is a subset of the Standard Generalized Markup Language (SGML) and has a very famous cousin named Hypertext Markup Language - or HTML. XML is, basically, a general purpose markup language for adding semantics to documents. This semantic is further formalized by either linking a Document Type Definition (DTD), or an XML Schema (XSL) to the document. A large advantage of XML is that it, like HTTP, is a widely used standard, and most programming languages have libraries for using XML without having to parse everything. XML - being a generalized language - has the further advantage that you can build new standards as XML applications and inherit the well-formedness\footnote{Grammatically and semantically correct.} property of of XML itself, and the general XML parser can used.

\item[WSDL] The \textbf{W}eb \textbf{S}ervice \textbf{D}efinition \textbf{L}anguage is an XML application introducing formalism to web application via definitions of ports and messages. A WSDL document can be retrieved from a server

\item[SOAP] \textbf{S}imple \textbf{O}object \textbf{A}ccess \textbf{P}rotocol is another XML application that provided a  set of encapsulation rules for messages sent via web services.

\item[BEPL] \textbf{B}usiness \textbf{P}rocess \textbf{E}xecution \textbf{L}anguage is also an XML application designed for formalizing business processes as well as executing them. BPEL is flow-oriented and and concentrates on message flow, control flow, data flow and the fault and exception handling when things do not go right. It also orchestrates the different flows between services -- called partners -- by becoming a central point of authority. It is layered atop of WSDL and can be visualized graphically.

\item[UDDI/Service discovery] \textbf{U}niversal \textbf{D}escription, \textbf{D}iscovery and \textbf{I}ntegration is a registry service for centralizing information about web services, and who provides provides them using XML in the form of WSIL\footnote{Web Services Inspection Language} documents. The UDDI is a also directory service providing information about the companies as well. The service providers are responsible for registering themselves into the registry, along with their WSDL specification.

\item[REST] The concept \textbf{RE}presentational \textbf{S}tate \textbf{T}ransfer covers the architecture defined by Roy Fielding and is specific application to HTTP (1.1) and centralized around the concept of resources and representations\footnote{Typically states of resources}. Resources are accessed and changed by ordinary HTTP methods: GET, PUT, POST, DELETE and standard HTTP 1.1 response codes are used to explain to the caller how the request went. For instance, a DELETE operation on a non-existing ID could return a 404 - Not Found, while it would return 200 - OK if all went well. There exists no (standardized) formalization language to describe RESTful web services, like there does with WSDL.

\item[JSON] \textbf{J}ava\textbf{S}cript \textbf{O}bject \textbf{N}otation is an object serialization language that is human-readable -- much like XML. One of the big advantages of JSON over XML is, however, that grammer, and thus effectively, the parser is very simple. It is also less verbose than XML and therefore has less overhead. It has draft support for schemas, and has currently no general way of validating data, or enforcing constraints. Due to its simple nature, it doesn't support any other types than the ones that are built-in. JSON is typically used in web applications, and in strong collaboration with the REST paradigm.

%TODO Find reference to XML book

\end{description}As all of the above technologies are standardized, interoperability becomes less of a hassle than if every protocol should be implemented from scratch. Using formalized languages -- such as WSDL, SOAP and BEPL -- provides additional benefits, such as client/server interface stub generation and implicit documentation.