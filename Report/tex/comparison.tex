{\setlength{\chapterfontsize}{26pt}
\chapter{Comparison of RESTful and SOAP/BPEL web services}
}
In general technologies covered in this project are standardized, interoperability is much less of a hassle than if every protocol should be implemented from scratch. Using formalized languages -- such as WSDL, SOAP and BEPL -- provides additional non-quantifiable benefits such as implicit documentation.\\\\
As we have implemented the TravelGood business process using both RESTful and SOAP/BPEL, a number of strengths and weaknesses have been experienced, which will be discussed in this chapter.

\section{Implementation}
Beginning with the tool chain, the easiest technology to get started with is REST. As a lot of concepts are and tool-dependant processes are linked to BEPL, you won't get very far without the tools. Whereas REST being ``just code'', all you really need is an editor and a compiler (and preferably a framework). Writing up the XML of BEPL without OpenESB is close to impossible other than for very simple examples.
\todo{Skriv noget om at RESTful er ret lige til mens BPEL er ond ond XML kode -  Noget i den retning?} 

\section{Understanding the implementation}
In general the basics of a RESTful web-service is easy to understand for anyone familiar with the HTTP verbs, however the business logic in a complicated RESTFul web-service can be hard to get an overview of. There is not a single place where a complete overview of the web-service can be found, but instead one is required to go through each of the resources in order to try to get an understanding of the processes. Usually a RESTful web-service will rely on some documentation describing the API. This documentation would have to be kept maintained according to the current implementation. 

The BPEL implementation is easier to read and understand for people with a non-technical background due to the visual representation of the processes. It directly shows the flow of progress in a web service. Furthermore the visual representation makes it possible to get an overview of all processes from a single place. It is, however, not as versatile as a RESTful implementation as only a limited set of operations are possible in the BPEL environment.

\section{Adaptability to change}
Changing the requirements is something that are realistic and a are part of almost every project, further from a point of view of maintainability it is a critical factor that the web-service adapts well to such changes.

For REST there is no standard metadata format which means that changes in the REST interface or the format returned will require services using the web-service to adapt to these changes as well. What usually happens today when a RESTful web-service in production is changed is that a new version of the web-service next to the old version, which will be kept running until all users has adapted to the new version. Several solutions exist for providing metadata for a RESTful web-service including \texttt{WADL}, but none of these has been widely adapted yet. Furthermore as a developer of a RESTful web-service new resources will have to be introduced.

In SOAP/BPEL the \texttt{WSDL} files makes it easier for services using the web-service to adapt to the changes since the interface is specified in the \texttt{WSDL}. The other web-services would still have to adapt for the changes made to this interface though, but these should be minimal if the same service is provided.

\section{Scalability and performance}
Both SOAP and RESTful web-services are deployed in large setups in the industry which indicates that both solutions are scaleable. The transmission overhead in a RESTful web-service is significantly lower especially when using JSON compared to SOAP/BPEL. When using advanced web-service technologies such as WS-Addressing and WS-Security more overhead is introduced on the bandwidth and parsing of the messages, however in practice does not have that much impact on the performance, as messages may be compressed during transfer\footnote{XML compresses well.}.\\\\
A thing that may have the most significant impact on SOAP/BPEL performance is the strict abstraction layer, which removes (or at least makes it harder) the possibilities to use classic optimization strategies such as caching proxies. This is due to the fact that is not part of the WS-* scope as must thus be handled \emph{ad hoc} with; either a separate BEPL caching process - or outside the scope of the application. RESTful is by-design built for distributed, proxyable (and thus cacheable) operation - effectively making scalability a natural part of the development process.

